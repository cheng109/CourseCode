\documentclass{article}

% Packages required to support encoding
\usepackage{ucs}
\usepackage[utf8x]{inputenc}
\usepackage{graphicx} 
% Packages required by code

% Packages always used
\usepackage{listings}
\usepackage{hyperref}
\usepackage{xspace}
\usepackage[usenames,dvipsnames]{color}
\hypersetup{colorlinks=true,urlcolor=blue}


\usepackage[framed,numbered,autolinebreaks,useliterate] {mcode}
\newcommand{\BigO}[1]{\ensuremath{\operatorname{O}\bigl(#1\bigr)}}

\input{preamble.tex}

\title{Homework}



\begin{document} 



\hypertarget{problem_0_homework_checklist_2}{}
\subsection*{{Problem 0: Homework checklist}}
\label{problem_0_homework_checklist_2}

\checkmark	I didn't talk with any one about this homework. \newline
\checkmark 	Source-code are included at the end of this document. 


\hypertarget{problem_0_homework_checklist_2}{}
\subsection*{{Problem 1: The Cholesky Factorization}}
\label{}
\begin{enumerate} 
\item 
Here is my implementation of Cholesky decomposition in Matlab: 
\begin{lstlisting}
function L = cholesky(A)
n = size(A); 
n = n(1); 
L = zeros(n,n);
%L(1,1) = sqrt(A(1,1)); 
for i=(1:n)
    sum1 =0;     
    for j=(1:i) 

        if i==j
            L(j,j) = sqrt(A(j,j)-sum1); 
        else 
            sum2 = 0; 
            for k = (1:j-1) 
                sum2 = sum2 + L(i,k)*L(j,k); 
            end 
            
            L(i,j) = 1/L(j,j)*(A(i,j)-sum2); 
        end 
        sum1= sum1 + L(i,j)^2;   
    end 
end 
end 
\end{lstlisting} 
\item 
Cholesky factorization is unique if $\mA$ is positive definite and the decomposition need not be unique when $\mA$ is positive semidefinite.  \\
Proof: \\
Get the LU decomposition of $\mA=\mL\mU$,  and $u_{11}>0, u_{22}>0, ..., u_{mm}>0 $ 
\begin{align*} 
\mU &=\bmat{u_{11} & u_{12} & ... & u_{1m} \\ 0 & u_{22} & ... & u_{2m} \\ 0 & 0 & ... & ... \\ 0 & 0 & ... & u_{mm} }\\
& = \bmat{u_{11} & 0 & ... & 0 \\ 0 & u_{22} & ... & 0 \\ 0 & 0 & ... & ... \\ 0 & 0 & ... & u_{mm}} \bmat{1 & v_{12} & ... & v_{1m} \\ 0 & 1 & ... & v_{2m} \\ 0 & 0 & ... & ... \\ 0 & 0 & ... & 1}=\mT\mW
\end{align*} 
Since $u_{11} >0, ..., u_{mm}>0 $,  we can get $\mT=\mD^2$, where \begin{align*} \mD = \bmat{\sqrt{u_{11}} & 0 & ... & 0 \\ 0 & \sqrt{u_{22}} & ... & 0 \\ 0 & 0 & ... & ... \\ 0 & 0 & ... & \sqrt{u_{mm}}} \\
\mA = \mL\mU = \mL\mD_2\mW \end{align*}

Because $\mA_*=\mA$ and the LU factorization is unique, then $\mL=\mW^*$. Therefore the Cholesky factorization is unique if $\mA$ is positive definite.  \\
Reference: http://math.utoledo.edu/~mtsui/4350sp08/homework/Lec23.pdf \\


\item 
I computed the norm of difference of my Cholesky result and Matlab $chlo$: \\
\begin{align*}   1.9110e-13   \end{align*} 


\item 
For each size of the matrices I repeated 10 times. The comparison result is shown.   Obviously the the Cholesky factorization has better performance as the growth of matrix size. 
\begin{figure} 
\centering 
\includegraphics[width=0.8\textwidth]{luVScholesky}
\caption{Performance comparison between LU decomposition and Cholesky factorization}
\end{figure}

\item 
Since $\mA$ is positive definite, and assume an vector $\vv = \bmat{x_0\\ \vx}, \\ $then \\
\begin{align*} 
\vv_T\mA\vv>0 \\
\bmat{x_0 & \vx}\bmat{\alpha & \vb_T\\ \vb & \mC} \bmat{x_0\\ \vx} = \alpha x_0^2 + x_0(\vx^T\vb+\vb^T\vx) + \vx^T\mC\vx >0  
\end{align*}
Then 
\begin{align*}
\vx^T(\mC-\frac{\vb\vb^T}{\alpha})\vx &= \vx^T\mC\vx - \frac{\vx_T\vb\vb^T\vx}{\alpha} \\
& > -\alpha x_0^2 - x_0(\vx^T\vb+\vb^T\vx) - \frac{\vx^T\vb\vb^T\vx}{\alpha} \\
& = -\frac{1}{\alpha} \left[x_0^2 + \frac{\vx^T\vb+\vb^T\vx)x_0}{\alpha}+\frac{\vx^T\vb\vb^T\vx}{\alpha^2}\right]\\
& = -\frac{1}{\alpha}\left(x_0+\frac{\vx^T\vb}{\alpha}\right)\left(x_0+\frac{\vb^T\vx}{\alpha}\right) \\
& =  -\frac{1}{\alpha}\left\|x_0+\frac{\vb^T\vx}{\alpha}\right\|^2 
\end{align*} 
Because $\vx^T(\mC-\frac{\vb\vb^T}{\alpha})\vx > -\frac{1}{\alpha}\left\|x_0+\frac{\vb^T\vx}{\alpha}\right\|^2$ holds for any vector $\vx$. Therefore \begin{align} 
\vx^T(\mC-\frac{\vb\vb^T}{\alpha})\vx > 0 
\end{align} 
So $\mC-\frac{\vb\vb^T}{\alpha}$ is positive definite. 

\item  
The base case is that $C$ is a $1\times1$. 

\item 
In the Matlab code, we need check positiveness before take square root.  The modified code is here:  
\begin{lstlisting} 
function L = cholesky(A)
n = size(A); 
n = n(1); 
L = zeros(n,n);
%L(1,1) = sqrt(A(1,1)); 
for i=(1:n)
    sum1 =0;     
    for j=(1:i) 

        if i==j
            if A(j,j)-sum1<=0
                return       % check whether A is positive definite. 
            else
                L(j,j) = sqrt(A(j,j)-sum1); 
            end 
        else 
            sum2 = 0; 
            for k = (1:j-1) 
                sum2 = sum2 + L(i,k)*L(j,k); 
            end 
            
            L(i,j) = 1/L(j,j)*(A(i,j)-sum2); 
        end 
        sum1= sum1 + L(i,j)^2;   
    end 
end 
end 
\end{lstlisting} 
After testing, the matrix for Poisson's equation is not positive definite or negative definite.  \\ 

\end{enumerate}

\hypertarget{}{}
\subsection*{{Problem 2: Stability analysis}}
\label{}
\begin{enumerate} 
\item 
\begin{align*} 
V &= \frac{1}{n-1}\sum_{i=1}^n(x_i-\frac{1}{n}x_i)^2 \\
& = \frac{1}{n-1}\left[\sum_{i=1}^nx_i^2 - \frac{1}{n}\left(\sum_{i=1}^nx_i\right)^2\right]
\end{align*}
The condition number is defined as : \\ 
\begin{align*} 
\kappa = lim_{\epsilon \to 0}sup\left[\frac{|V(x)-V(x+\Delta x)|}{\epsilon V(x)}: \|\Delta x\|_2 \leq \epsilon\|x\|_2\right]
\end{align*}  
Then \\ 
\begin{align*} 
\kappa = 2\frac{\|x\|_2}{\sqrt{(n-1)V(x)}}
\end{align*} 
\item 
This is not appropriate.  I will use $ q=100 and s =2 $ to do a test. 
\begin{lstlisting} 
q = 500; 
s = 7; 
z = zeta(s); 
h = z-sum((1:(q-1)).^(-s)) 
\end{lstlisting} 
Hurwitz zeta function: 
\begin{align*} H(s,q) &= \sum_{n=0}^\infty\frac{1}{(q+n)^s} \\
& = \frac{1}{q^s}+ \frac{1}{(q+1)^s} + ... +  \frac{1}{n^s}\\
\end{align*} 
Riemann zeta function: 
\begin{align*}
R(s) &= \sum_{n=1}^\infty\frac{1}{n^s}\\
& = \frac{1}{1^s} + \frac{1}{2^s} + ... + \frac{1}{n^s}\\
& = \frac{1}{1^s} + \frac{1}{2^s} + ... +\frac{1}{(q-1)^2} + \frac{1}{q^s}+ \frac{1}{(q+1)^s} + ... +  \frac{1}{n^s} \\
& = \frac{1}{1^s} + \frac{1}{2^s} + ... +\frac{1}{(q-1)^2} + H(s,q) \\
\end{align*} 
Output from Matlab is:  \begin{align*} 2.2204e-15 \end{align*} \\
But the true answer from Wolfram Alpha is \\\begin{align*} 1.073081599928320 \times 10^{-17} \end{align*}
Here is the explanation:  when you calculate the Riemann zeta function,  the terms with larger $n$ will round off because they are relative smaller than terms with smaller $n$.  The leading terms
\begin{align*} 
\frac{1}{1^s} + \frac{1}{2^s} + ... +\frac{1}{(q-1)^2} 
\end{align*} 
is dominant in $R(s)$. But those are the terms we need for Hurwitz zeta with large $q$.  Once you round off them, there would be large errors. \\
\end{enumerate} 

\hypertarget{}{}
\subsection*{{Problem 3: Backwards stability}}
\label{}

\begin{align*}
fl(x_i) = x_i(1+\epsilon_{ix}) \\
fl(y_i) = y_i(1+\epsilon_{iy}) \\
\end{align*} 
where $|\epsilon_{ix}|, |\epsilon_{iy}|\leq\epsilon_{machine} $ \\
\begin{align*}
\alpha &=\vx^T\vy \\
& = \bmat{x_1 & x_2 & .. & x_n}\bmat{y_1 \\ y_2 \\ .. \\ y_n} \\
& = \sum_{i=1}^nx_i y_i \\ 
\end{align*} 
Then \begin{align*} 
fl(\vx^T\vy) & = \sum_{i=1}^nfl(x_i)\times fl( y_i) \\
& = \left[fl(x_1)\times fl(y_1)+ fl(x_2)\times fl(y_2) + ... + fl(x_n)\times fl(y_n)\right] (1+\epsilon_{addition})\\
& = \left[x_1y_1(1+\epsilon_{1x})(1+\epsilon_{1y}) + ... + x_ny_n(1+\epsilon_{nx})(1+\epsilon_{ny})\right](1+\epsilon_{addition}) \\ 
& = \left[x_1y_1 + x_2y_2 + ... +x_ny_n\right](1+\epsilon_1)(1+\epsilon_2)(1+\epsilon_3)\\
& = \left[x_1y_1 + x_2y_2 + ... +x_ny_n\right](1+\BigO{\epsilon_{machine}})
\end{align*}
Therefore the inner product is backwards stable. \\

\end{document}
