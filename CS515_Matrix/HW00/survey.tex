\documentclass{article}

% Packages required to support encoding
\usepackage{ucs}
\usepackage[utf8x]{inputenc}

% Packages required by code


% Packages always used
\usepackage{hyperref}
\usepackage{xspace}
\usepackage[usenames,dvipsnames]{color}
\hypersetup{colorlinks=true,urlcolor=blue}


\usepackage{geometry}
\geometry{letterpaper,textwidth=350pt,textheight=680pt,tmargin=60pt,
            left=72pt,footskip=24pt,headsep=18pt,headheight=14pt}
\usepackage{amsmath}
\usepackage{amssymb}
\usepackage{textcase}
\usepackage{soul}

\newcommand{\mat}[1]{\boldsymbol{#1}}\renewcommand{\vec}[1]{\boldsymbol{\mathrm{#1}}}
\newcommand{\vecalt}[1]{\boldsymbol{#1}}

\newcommand{\conj}[1]{\overline{#1}}

\newcommand{\normof}[1]{\|#1\|}
\newcommand{\onormof}[2]{\|#1\|_{#2}}

\newcommand{\itr}[2]{#1^{(#2)}}
\newcommand{\itn}[1]{^{(#1)}}

\newcommand{\eps}{\varepsilon}
\newcommand{\kron}{\otimes}

\DeclareMathOperator{\diag}{diag}
\DeclareMathOperator{\trace}{trace}
\DeclareMathOperator{\tvec}{vec}

\newcommand{\prob}{\mathbb{P}}
\newcommand{\probof}[1]{\prob\left\{ #1 \right\}}

\newcommand{\pmat}[1]{\begin{pmatrix} #1 \end{pmatrix}}
\newcommand{\bmat}[1]{\begin{bmatrix} #1 \end{bmatrix}}
\newcommand{\spmat}[1]{\left(\begin{smallmatrix} #1 \end{smallmatrix}\right)}
\newcommand{\sbmat}[1]{\left[\begin{smallmatrix} #1 \end{smallmatrix}\right]}

\newcommand{\RR}{\mathbb{R}}
\newcommand{\CC}{\mathbb{C}}

\providecommand{\eye}{\mat{I}}
\providecommand{\mA}{\ensuremath{\mat{A}}}
\providecommand{\mB}{\ensuremath{\mat{B}}}
\providecommand{\mC}{\ensuremath{\mat{C}}}
\providecommand{\mD}{\ensuremath{\mat{D}}}
\providecommand{\mE}{\ensuremath{\mat{E}}}
\providecommand{\mF}{\ensuremath{\mat{F}}}
\providecommand{\mG}{\ensuremath{\mat{G}}}
\providecommand{\mH}{\ensuremath{\mat{H}}}
\providecommand{\mI}{\ensuremath{\mat{I}}}
\providecommand{\mJ}{\ensuremath{\mat{J}}}
\providecommand{\mK}{\ensuremath{\mat{K}}}
\providecommand{\mL}{\ensuremath{\mat{L}}}
\providecommand{\mM}{\ensuremath{\mat{M}}}
\providecommand{\mN}{\ensuremath{\mat{N}}}
\providecommand{\mO}{\ensuremath{\mat{O}}}
\providecommand{\mP}{\ensuremath{\mat{P}}}
\providecommand{\mQ}{\ensuremath{\mat{Q}}}
\providecommand{\mR}{\ensuremath{\mat{R}}}
\providecommand{\mS}{\ensuremath{\mat{S}}}
\providecommand{\mT}{\ensuremath{\mat{T}}}
\providecommand{\mU}{\ensuremath{\mat{U}}}
\providecommand{\mV}{\ensuremath{\mat{V}}}
\providecommand{\mW}{\ensuremath{\mat{W}}}
\providecommand{\mX}{\ensuremath{\mat{X}}}
\providecommand{\mY}{\ensuremath{\mat{Y}}}
\providecommand{\mZ}{\ensuremath{\mat{Z}}}
\providecommand{\mLambda}{\ensuremath{\mat{\Lambda}}}
\providecommand{\mPbar}{\bar{\mP}}

\providecommand{\ones}{\vec{e}}
\providecommand{\va}{\ensuremath{\vec{a}}}
\providecommand{\vb}{\ensuremath{\vec{b}}}
\providecommand{\vc}{\ensuremath{\vec{c}}}
\providecommand{\vd}{\ensuremath{\vec{d}}}
\providecommand{\ve}{\ensuremath{\vec{e}}}
\providecommand{\vf}{\ensuremath{\vec{f}}}
\providecommand{\vg}{\ensuremath{\vec{g}}}
\providecommand{\vh}{\ensuremath{\vec{h}}}
\providecommand{\vi}{\ensuremath{\vec{i}}}
\providecommand{\vj}{\ensuremath{\vec{j}}}
\providecommand{\vk}{\ensuremath{\vec{k}}}
\providecommand{\vl}{\ensuremath{\vec{l}}}
\providecommand{\vm}{\ensuremath{\vec{l}}}
\providecommand{\vn}{\ensuremath{\vec{n}}}
\providecommand{\vo}{\ensuremath{\vec{o}}}
\providecommand{\vp}{\ensuremath{\vec{p}}}
\providecommand{\vq}{\ensuremath{\vec{q}}}
\providecommand{\vr}{\ensuremath{\vec{r}}}
\providecommand{\vs}{\ensuremath{\vec{s}}}
\providecommand{\vt}{\ensuremath{\vec{t}}}
\providecommand{\vu}{\ensuremath{\vec{u}}}
\providecommand{\vv}{\ensuremath{\vec{v}}}
\providecommand{\vw}{\ensuremath{\vec{w}}}
\providecommand{\vx}{\ensuremath{\vec{x}}}
\providecommand{\vy}{\ensuremath{\vec{y}}}
\providecommand{\vz}{\ensuremath{\vec{z}}}
\providecommand{\vpi}{\ensuremath{\vecalt{\pi}}}

\sodef\allcapsspacing{\upshape}{0.15em}{0.65em}{0.6em}%

\makeatletter
\def\maketitle{%
\par
\hrule height 0.75pt\vspace{1ex}
\par\noindent
\begin{minipage}{0.5\textwidth}
\scshape
purdue university $\cdot$ CS 580 \\
Introduction to the Analysis of Algorithms
\end{minipage}
\begin{minipage}{0.5\textwidth}
\raggedleft
\MakeTextUppercase{\allcapsspacing{\@title}}\\[0.2ex]
\textit{\@author}\\[0.2ex]
\textit{\@date}
\end{minipage}
\par\vspace{1ex}
\hrule height 1pt
\vspace{2ex}
\par
}
\makeatother

\author{Jun Cheng}
\title{Lecture Notes}
% auto generate a title
\AtBeginDocument{\maketitle}


\title{Homework}



\begin{document} 
Please answer the following questions in complete sentences in a typed manuscript and submit the solution on blackboard by on August 28rd at 5pm.

\hypertarget{yourself_1}{}\subsubsection*{{Yourself}}\label{yourself_1}

\begin{enumerate}%
\item Please tell me about yourself: name, MS/PhD objective, adviser (if you have one), year in program, research area.

Jun Cheng,   PhD student in physics with concentration in computer science and engineering. This is my 4th year in astrophysics research area.  Advisor: John Peterson (physics) 


\item Why are you taking the class?

I am in Compute Science and Engineering program, and students in this program are required to take two or more course in in computer science. 

\end{enumerate}
\hypertarget{the_course_2}{}\subsubsection*{{The course}}\label{the_course_2}

\begin{enumerate}%
\item Please answer the collaboration policy survey on Piazza.


\item The homeworks will be a mix of examples, applications, coding, and theory. For instance, I might have a few easy ``{}practice''{} questions about solving small linear systems. Then I might have a multi-step application that develops a general problem such as\ldots{} ``{}figure out where people are moving and where they are likely to be in 2050''{} into a matrix algorithm. There will also be some coding work, such as ``{}write a program to solve a linear system using the LU decomposition without pivoting''{}. Finally, there will be a theory component to the homeworks. These problems will ask you to prove a matrix statement.

Do you find you learn better with any particular type of problem? If so, which one?

I prefer the multiple-step application. Because I can learn more from those problems. 

\item Would you be interested in extra credit opportunities that extend the homework questions in more difficult ways? For instance, making your implementations fast or in C++.

Yes.  In fact I prefer to use C++ because my target job is a quantitative analyst in financial area and C++ is highly required skills in this area.  Therefore I'd like to have more chances to improve my programming skills in C++. 



\item Would you be interested in sharing any of the matrix problems you encounter with the class in a 3-5 minute presentation?

I would like if have some. But not at this point. 


\item What have other professors done that you'{}ve found helps you learn?

Not yet. 


\end{enumerate}
\hypertarget{numerical_computing_software_3}{}\subsubsection*{{Numerical computing software}}\label{numerical_computing_software_3}

\begin{enumerate}%
\item Have you used Matlab before?

Yes. 

\item Have you used NumPy/SciPy before?

Yes, I used python a lot, but not so much experience with NumPy/SciPy

\item Have you used Julia before?

No. It is my first time to know it in your class. 

\item Have you used R before?

Yes, I used R in STAT526 at Purdue. 

\item Have you used Mathematica before?

Yes, I only used online version. 

\item Any other numerical computing packages?

No. 


\end{enumerate}
\hypertarget{the_course_4}{}\subsubsection*{{The course}}\label{the_course_4}

\begin{enumerate}%
\item Which of the topics from the syllabus are you most excited about?

I used Cholesky factorization in one of my previous class, and I learned that by myself.  

\item Anything missing from the syllabus you were hoping to learn about?

No any at this point. 


\end{enumerate}
\hypertarget{videos_5}{}\subsubsection*{{Videos}}\label{videos_5}

\begin{enumerate}%
\item Would you be interested in having access to the video taped lectures from last year? If so, should I provide them before or after the same lecture this year?

Yes, I think it would be better after the class. 

\end{enumerate}
\hypertarget{flipped_classroom_6}{}\subsubsection*{{Flipped classroom}}\label{flipped_classroom_6}

\begin{enumerate}%
\item Have you ever had a flipped classroom or flipped lecture?

No, never. 

\item If so, should we try this for a few lectures in CS515 this year? Why or why not?

I am very curious about this. I would like to have a try. 


\end{enumerate}

\end{document}
