\documentclass{article}

% Packages required to support encoding
\usepackage{ucs}
\usepackage[utf8x]{inputenc}

% Packages required by code


% Packages always used
\usepackage{hyperref}
\usepackage{xspace}
\usepackage[usenames,dvipsnames]{color}
\hypersetup{colorlinks=true,urlcolor=blue}


\usepackage{geometry}
\geometry{letterpaper,textwidth=350pt,textheight=680pt,tmargin=60pt,
            left=72pt,footskip=24pt,headsep=18pt,headheight=14pt}
\usepackage{amsmath}
\usepackage{amssymb}
\usepackage{textcase}
\usepackage{soul}

\newcommand{\mat}[1]{\boldsymbol{#1}}\renewcommand{\vec}[1]{\boldsymbol{\mathrm{#1}}}
\newcommand{\vecalt}[1]{\boldsymbol{#1}}

\newcommand{\conj}[1]{\overline{#1}}

\newcommand{\normof}[1]{\|#1\|}
\newcommand{\onormof}[2]{\|#1\|_{#2}}

\newcommand{\itr}[2]{#1^{(#2)}}
\newcommand{\itn}[1]{^{(#1)}}

\newcommand{\eps}{\varepsilon}
\newcommand{\kron}{\otimes}

\DeclareMathOperator{\diag}{diag}
\DeclareMathOperator{\trace}{trace}
\DeclareMathOperator{\tvec}{vec}

\newcommand{\prob}{\mathbb{P}}
\newcommand{\probof}[1]{\prob\left\{ #1 \right\}}

\newcommand{\pmat}[1]{\begin{pmatrix} #1 \end{pmatrix}}
\newcommand{\bmat}[1]{\begin{bmatrix} #1 \end{bmatrix}}
\newcommand{\spmat}[1]{\left(\begin{smallmatrix} #1 \end{smallmatrix}\right)}
\newcommand{\sbmat}[1]{\left[\begin{smallmatrix} #1 \end{smallmatrix}\right]}

\newcommand{\RR}{\mathbb{R}}
\newcommand{\CC}{\mathbb{C}}

\providecommand{\eye}{\mat{I}}
\providecommand{\mA}{\ensuremath{\mat{A}}}
\providecommand{\mB}{\ensuremath{\mat{B}}}
\providecommand{\mC}{\ensuremath{\mat{C}}}
\providecommand{\mD}{\ensuremath{\mat{D}}}
\providecommand{\mE}{\ensuremath{\mat{E}}}
\providecommand{\mF}{\ensuremath{\mat{F}}}
\providecommand{\mG}{\ensuremath{\mat{G}}}
\providecommand{\mH}{\ensuremath{\mat{H}}}
\providecommand{\mI}{\ensuremath{\mat{I}}}
\providecommand{\mJ}{\ensuremath{\mat{J}}}
\providecommand{\mK}{\ensuremath{\mat{K}}}
\providecommand{\mL}{\ensuremath{\mat{L}}}
\providecommand{\mM}{\ensuremath{\mat{M}}}
\providecommand{\mN}{\ensuremath{\mat{N}}}
\providecommand{\mO}{\ensuremath{\mat{O}}}
\providecommand{\mP}{\ensuremath{\mat{P}}}
\providecommand{\mQ}{\ensuremath{\mat{Q}}}
\providecommand{\mR}{\ensuremath{\mat{R}}}
\providecommand{\mS}{\ensuremath{\mat{S}}}
\providecommand{\mT}{\ensuremath{\mat{T}}}
\providecommand{\mU}{\ensuremath{\mat{U}}}
\providecommand{\mV}{\ensuremath{\mat{V}}}
\providecommand{\mW}{\ensuremath{\mat{W}}}
\providecommand{\mX}{\ensuremath{\mat{X}}}
\providecommand{\mY}{\ensuremath{\mat{Y}}}
\providecommand{\mZ}{\ensuremath{\mat{Z}}}
\providecommand{\mLambda}{\ensuremath{\mat{\Lambda}}}
\providecommand{\mPbar}{\bar{\mP}}

\providecommand{\ones}{\vec{e}}
\providecommand{\va}{\ensuremath{\vec{a}}}
\providecommand{\vb}{\ensuremath{\vec{b}}}
\providecommand{\vc}{\ensuremath{\vec{c}}}
\providecommand{\vd}{\ensuremath{\vec{d}}}
\providecommand{\ve}{\ensuremath{\vec{e}}}
\providecommand{\vf}{\ensuremath{\vec{f}}}
\providecommand{\vg}{\ensuremath{\vec{g}}}
\providecommand{\vh}{\ensuremath{\vec{h}}}
\providecommand{\vi}{\ensuremath{\vec{i}}}
\providecommand{\vj}{\ensuremath{\vec{j}}}
\providecommand{\vk}{\ensuremath{\vec{k}}}
\providecommand{\vl}{\ensuremath{\vec{l}}}
\providecommand{\vm}{\ensuremath{\vec{l}}}
\providecommand{\vn}{\ensuremath{\vec{n}}}
\providecommand{\vo}{\ensuremath{\vec{o}}}
\providecommand{\vp}{\ensuremath{\vec{p}}}
\providecommand{\vq}{\ensuremath{\vec{q}}}
\providecommand{\vr}{\ensuremath{\vec{r}}}
\providecommand{\vs}{\ensuremath{\vec{s}}}
\providecommand{\vt}{\ensuremath{\vec{t}}}
\providecommand{\vu}{\ensuremath{\vec{u}}}
\providecommand{\vv}{\ensuremath{\vec{v}}}
\providecommand{\vw}{\ensuremath{\vec{w}}}
\providecommand{\vx}{\ensuremath{\vec{x}}}
\providecommand{\vy}{\ensuremath{\vec{y}}}
\providecommand{\vz}{\ensuremath{\vec{z}}}
\providecommand{\vpi}{\ensuremath{\vecalt{\pi}}}

\sodef\allcapsspacing{\upshape}{0.15em}{0.65em}{0.6em}%

\makeatletter
\def\maketitle{%
\par
\hrule height 0.75pt\vspace{1ex}
\par\noindent
\begin{minipage}{0.5\textwidth}
\scshape
purdue university $\cdot$ CS 580 \\
Introduction to the Analysis of Algorithms
\end{minipage}
\begin{minipage}{0.5\textwidth}
\raggedleft
\MakeTextUppercase{\allcapsspacing{\@title}}\\[0.2ex]
\textit{\@author}\\[0.2ex]
\textit{\@date}
\end{minipage}
\par\vspace{1ex}
\hrule height 1pt
\vspace{2ex}
\par
}
\makeatother

\author{Jun Cheng}
\title{Lecture Notes}
% auto generate a title
\AtBeginDocument{\maketitle}


\title{Homework}



\begin{document} 
\hypertarget{homework_1b_1}{}\section*{{Homework 1b}}\label{homework_1b_1}

Please answer the following questions in complete sentences in a clearly, typed prepared manuscript and submit the solution by the due date on Blackboard (Friday, September 11th, 2015, 5pm)

\hypertarget{problem_0_homework_checklist_2}{}\subsection*{{Problem 0: Homework checklist}}\label{problem_0_homework_checklist_2}

\begin{itemize}%
\item Please identify anyone, whether or not they are in the class, with whom you discussed your homework. This problem is worth 1 point, but on a multiplicative scale.


\item Make sure you have included your source-code and prepared your solution according to the most recent Piazza note on homework submissions.



\end{itemize}
\hypertarget{problem_1_prove_or_disprove_3}{}\subsection*{{Problem 1: Prove or disprove}}\label{problem_1_prove_or_disprove_3}

For the following questions, either prove that the statement is correct, or show a counter-example.

\begin{enumerate}%
\item The product of two diagonal matrices is diagonal.


\item The product of two upper triangular matrices is upper triangular


\item The product of two symmetric matrices is symmetric.


\item The product of two orthogonal matrices is orthogonal.


\item The product of two square, full rank matrices is full rank



\end{enumerate}
\hypertarget{problem_2_4}{}\subsection*{{Problem 2}}\label{problem_2_4}

There are a tremendous number of matrix norms that arise. An interesting class are called the \emph{orthgonally invariant norms}. Norms in this class satisfy:

\begin{displaymath}
\normof{\mA} = \normof{\mU \mA \mV}
\end{displaymath}
for \emph{square orthogonal matrices} $\mU$ and $\mV$. Recall that a square matrix is orthogonal when $\mU^T \mU = \mI$, i.e. $\mU^{-1} = \mU^T$.

Show that $\normof{ \mA }_2$ is orthogonally invariant.

\hypertarget{problem_3_5}{}\subsection*{{Problem 3}}\label{problem_3_5}

Consider the following function:

\begin{displaymath}
f(\mA) = \max_{i,j} |A_{i,j}|.
\end{displaymath}
\begin{enumerate}%
\item Show that $f$ is a matrix norm. (Very easy!)


\item Show that $f$ does not satisfy the sub-multiplicative property.


\item Show that there exists $\sigma > 0$ such that:

\begin{displaymath}
g(\mA) = \sigma f(\mA)
\end{displaymath}
is a sub-multiplicative matrix-norm.


\item \textbf{Extra tough problem for the adventurous! Not graded.} This problem has a relatively easy proof related to something we saw in class. But making it fully formal requires a few technicalities that are easy to get tripped up on. Now let $\normof{\mA}$ be an arbitrary matrix norm. Show that there exists $\sigma > 0$ such that $h(\mA) = \sigma \normof{\mA}$ is a sub-multiplicative matrix-norm.



\end{enumerate}
\hypertarget{problem_4_6}{}\subsection*{{Problem 4}}\label{problem_4_6}

Let $\normof{\mA}$ be a matrix norm and let $\vk \not= 0$ be a real-valued vector. Consider the function:

\begin{displaymath}
f(\vx) = \normof{ \vx \vk^T }.
\end{displaymath}
\begin{enumerate}%
\item Show that $f$ is a vector norm.


\item Show that if $\normof{\mA}$ is a sub-multiplicative matrix norm, then the vector norm $f$ is \emph{consistent} with the matrix norm. That is:

\begin{displaymath}
f(\mA \vx) \le \normof{\mA} f(\vx).
\end{displaymath}


\end{enumerate}
\hypertarget{problem_5_choice_1_7}{}\subsection*{{Problem 5 (Choice 1)}}\label{problem_5_choice_1_7}

Note that you only have to do one of the two choices for problem 5.

In class we mentioned that the function

\begin{displaymath}
f(\vx) = \text{ sum of two largest entries in $\vx$ by absolute value }
\end{displaymath}
is a vector norm. Use a computer to prepare a 3d plot of the unit-ball for this norm where $\vx \in \RR^{3}$. We may award up to 5 bonus points for an excellent graphic.

\hypertarget{problem_5_choice_2_8}{}\subsection*{{Problem 5 (Choice 2)}}\label{problem_5_choice_2_8}

Note that you only have to do one of the two choices for problem 5.

Let $f(\vx)$ be a vector norm. If $\vx \in \RR^{1}$ (that is, $x$ is a scalar), show that $f(x) = \alpha |x|$, where $|x|$ is the absolute value function.


\end{document}
