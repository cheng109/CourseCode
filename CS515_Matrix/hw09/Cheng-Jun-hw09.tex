\documentclass{article}

% Packages required to support encoding
\usepackage{ucs}
\usepackage[utf8x]{inputenc}
\usepackage{graphicx} 
% Packages required by code

% Packages always used
\usepackage{listings}
\usepackage{hyperref}
\usepackage{xspace}
\usepackage[usenames,dvipsnames]{color}
\hypersetup{colorlinks=true,urlcolor=blue}


\usepackage[framed,numbered,autolinebreaks,useliterate] {mcode}


\input{preamble.tex}

\title{Homework}



\begin{document} 



\hypertarget{problem_0_homework_checklist_2}{}
\subsection*{{Problem 0: Homework checklist}}
\label{}

\checkmark	I asked RUN CHEN about Lanczos. \newline
\checkmark 	Source-code are included at the end of this document. 

\hypertarget{}{}
\subsection*{{Problem 1:  Compare CG as described in class to the standard CG iteration}}
\label{}


\begin{enumerate} 
\item 
The CGL code is shown as below: \\
\begin{verbatim} 
function [res, x] = CGL(A, b, n, tol, num)
T = zeros(n, n);
v = zeros(n, n+1);
res = zeros(n);
alpha = zeros(n, 1); 
beta = zeros(n+1,1); 
beta(1) = norm(b); 
v0 = zeros(n, 1);
v_temp = b/beta(1);
v(1:n, 1) = v_temp;
  
for j = 1 : num
    wj = A*v_temp;
    alpha(j)  = transpose(v_temp)*wj;
    T(j, j) = alpha(j); 
    wj = wj - beta(j)*v0 -alpha(j) *v_temp;
    beta(j+1) = norm(wj);
    v0 = v_temp;
    v_temp = wj/beta(j+1);
    if j ~= n
        T(j, j+1) = beta(j+1);
        T(j+1, j) = beta(j+1);
        v(1:n, j+1) = v_temp;
    end
  
    e1 = zeros(j, 1); 
    e1(1) = 1; 
    z = T(1:j, 1:j)\(norm(b)*e1); 
    x = v(1:n, 1:j)*z;
    res(j) =norm(beta(j+1)*z)-norm(b);
    if res(j) < tol
        break
    end
end

return
\end{verbatim} 

\item 
Plots are shown below:  
\begin{figure} 
\includegraphics[width=0.7\textwidth]{problem1_2_cg}
\centering
\caption{Problem 1.2. Residual vs iteration for cg.m code} 
\end{figure} 

\begin{figure} 
\includegraphics[width=0.7\textwidth]{problem1_2_mine}
\centering
\caption{Problem 1.2.  Residual vs iteration for CGL.m code} 
\end{figure} 

\item
When the tolerance is $10^{-8}$ , the size of residual is 50.    This is half of the size $n=100$. 


\end{enumerate} 

\hypertarget{}{}
\subsection*{{Problem 2: Orthogonality of Lanczos }}
\label{}

\begin{enumerate} 
\item 
I should find that the quantity is increasing converge to some value, and it doesn't depend on the starting vector.  Actually I find that. 
\begin{figure} 
\includegraphics[width=0.7\textwidth]{problem2_1_fix}
\centering
\caption{Problem 2.1.  A fixed starting point} 
\end{figure} 
\begin{figure} 
\includegraphics[width=0.7\textwidth]{problem2_1_random}
\centering
\caption{Problem 2.1.  A random starting point} 
\end{figure} 

\item 
The plots are shown below: \\
\begin{figure} 
\includegraphics[width=0.7\textwidth]{problem2_2_1_fixed}
\centering
\caption{Problem 2.2.  A fix starting point of $log(|\vv_1^T\vv_k|+10^{-20})$} 
\end{figure} 
\begin{figure} 
\includegraphics[width=0.7\textwidth]{problem2_2_1_random}
\centering
\caption{Problem 2.2.  A random starting point of $ log(|\vv_1^T\vv_k|+10^{-20})$} 
\end{figure} 


\begin{figure} 
\includegraphics[width=0.7\textwidth]{problem2_2_2_fixed}
\centering
\caption{Problem 2.2.  A fix starting point of $log(|\vv_{k-2}^T\vv_k|+10^{-20})$} 
\end{figure} 
\begin{figure} 
\includegraphics[width=0.7\textwidth]{problem2_2_2_random}
\centering
\caption{Problem 2.2.  A random starting point of $ log(|\vv_{k-2}^T\vv_k|+10^{-20})$} 
\end{figure} 



\item 
Because matrix $\mT$ is a tridiagonal matrix, then $\beta_{3,1}$ is supposed to be zero.  But I got $\beta_{3, 1} = 1.4726e^{-8}$

\item 
Plots are shown below: 
\begin{figure} 
\includegraphics[width=0.7\textwidth]{problem2_4_fixed}
\centering
\caption{Problem 2.4.  A fix starting point.} 
\end{figure} 
\begin{figure} 
\includegraphics[width=0.7\textwidth]{problem2_4_random}
\centering
\caption{Problem 2.4.  A random starting point. } 
\end{figure} 




\end{enumerate} 

\end{document}
