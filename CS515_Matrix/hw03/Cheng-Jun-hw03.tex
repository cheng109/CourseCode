\documentclass{article}

% Packages required to support encoding
\usepackage{ucs}
\usepackage[utf8x]{inputenc}
\usepackage{graphicx} 
% Packages required by code

% Packages always used
\usepackage{listings}
\usepackage{hyperref}
\usepackage{xspace}
\usepackage[usenames,dvipsnames]{color}
\hypersetup{colorlinks=true,urlcolor=blue}


\usepackage[framed,numbered,autolinebreaks,useliterate] {mcode}


\input{preamble.tex}

\title{Homework}



\begin{document} 



\hypertarget{problem_0_homework_checklist_2}{}
\subsection*{{Problem 0: Homework checklist}}
\label{problem_0_homework_checklist_2}

\checkmark	I didn't talk with any one about this homework. \newline
\checkmark 	Source-code are included at the end of this document. 

\hypertarget{problem_0_homework_checklist_2}{}
\subsection*{{Problem 1}}
\label{problem_0_homework_checklist_2}

\begin{enumerate}
\item $\bmat{0 & 5 \\ 0 & 0}  = \bmat{1&0\\0&1} \bmat{5&0\\0 &0} \bmat{0&1\\-1 &0}$


\item $\bmat{5 & 0 \\ 2 & 0}$


\item $\bmat{5 & -5 \\ 2 & -2 \\ 0 & 0}$


\item $\bmat{2 & 0 \\ 0 & -1}$

\end{enumerate}



\hypertarget{problem_0_homework_checklist_2}{}
\subsection*{{Problem 2: }}
\label{problem_0_homework_checklist_2}

\begin{enumerate}

\item 
For example, The matrix \\
\begin{align} 
\mA = \bmat{1 &1\\ 2& 2} 
\end{align}
has two different  decomposition.  
But in both cases they have the same singular value. \\
\item 

\item 
Suppose $\mA$ is a matrix with eigenvalues $\{\sigma_i\}$\\
The SVD of $\mA$ is $\mA=\mU \Sigma\mV^T $ where $\mU$ and $\mV$ are both orthogonal matrix. 

\begin{flalign} 
\mA^{-1} &= (\mU \Sigma\mV^T)^{-1}\\
&=(\mV^T)^{-1}\Sigma^{-1}\mU^{-1}\\
&=(\mV^T)^T\Sigma^{-1}\mU^T\\
&=\mV\Sigma^{-1}\mU^T &&
\end{flalign}
Then  $\mA^{-1}$ is a matrix with eigenvalues $\{\frac{1}{\sigma_i}\}$
\begin{flalign} 
\|\mA^{-1}\| &= max\{\frac{1}{\sigma_i}\} \\
& = \frac{1}{\sigma_{min}} \\
\end{flalign}

\item 
The SVD of $\mQ$ is $\mQ=\mU \Sigma\mV^T=\mQ \mI\mI^T $ where $\mI$ is an identity matrix and it is also an orthogonal matrix. \\
In particular, the singular value are all $1$. 

\end{enumerate}

\hypertarget{problem_0_homework_checklist_2}{}
\subsection*{{Problem 3: }}
\label{}

\begin{enumerate} 
\item 
If $m<n$, we can use $\mA^{T}= (\mU\Sigma\mV^T)^{T} = \mV\Sigma\mU^T$ is still SVD. 

\item 
$\mA = diag(\sigma_1, \sigma_2, ...\sigma_n, 0, ...) $\\
Then the best diagonal rank $k$ approximation to $\mA$ is \\
 $\mA_k =diag(\sigma_1, \sigma_2, ...\sigma_k, 0, ...)$ \\
 We aussume $k<n$ since we want to do the lower rank approximation. \\
 Then  
 \begin{flalign}
 \|\mA-\mA_k\|_2  &= \|diag(\sigma_1, \sigma_2, ...\sigma_n, 0, ...) - diag(\sigma_1, \sigma_2, ...\sigma_k, 0, ...) \|_2\\
&= \| diag(0 , .., \sigma_{k+1}, ...\sigma_n, 0, ...) \|_2 \\
&= max\{\sigma_{k+1}, ...\sigma_n \}
\end{flalign}
The small least answer could be $\sigma_n$ which is the smallest singular value of $\mA$. \\
\item 

\begin{flalign} 
\|\mA-\mA_n\| &= \|\sum_{i=1}^{n}\sigma_i\vu_i\vv_i^T-\sum_{i=1}^{n}\sigma_i\vu_i\vv_i^T\|\\
& = 0\\
\end{flalign}



\item 
\begin{flalign} 
\|\mA-\mA_R\| &= \|\sum_{i=1}^{n}\sigma_i\vu_i\vv_i^T-\sum_{i=1}^{R}\sigma_i\vu_i\vv_i^T\|\\
& = \|\sum_{i=R+1}^{n}\sigma_i\vu_i\vv_i^T\|\\
& = max\{\sigma_i\}\\
\end{flalign}
When they have the same set of singular values, Then 
\begin{align} \|\mA-\mA_R\| = 0\\ 
\end{align}

\item 
\begin{flalign} 
\|\mA-\mA_k\| &= \|\sum_{i=1}^{n}\sigma_i\vu_i\vv_i^T-\sum_{i=1}^{k}\sigma_i\vu_i\vv_i^T\|\\
& = \|\sum_{i=k+1}^{n}\sigma_i\vu_i\vv_i^T\|\\
& = max\{\sigma_{k+1}, ..., \sigma_{n}\}\\
& = \sigma_{k+1}
\end{flalign}
$\sigma_{k+1}$ is the largest singular value of $\mA-\mA_k$ \\
\end{enumerate} 

\hypertarget{problem_0_homework_checklist_2}{}
\subsection*{{Problem 4: }}
\label{}
\begin{enumerate} 
\item 
The data is stored as a 3 dimension array $256\times 1100 \times 10$. \\
 For each digit, there are 1100 images.  Those images include 256 pixels with shape $16 by 16 $ \\ And the sample images are shown as below. \\
 \begin{figure}
 \includegraphics[width=0.25\textwidth, height=0.25\textwidth]{pic0}
\includegraphics[width=0.25\textwidth, height=0.25\textwidth]{pic1}   \includegraphics[width=0.25\textwidth, height=0.25\textwidth]{pic2}
\includegraphics[width=0.25\textwidth, height=0.25\textwidth]{pic3}
\includegraphics[width=0.25\textwidth, height=0.25\textwidth]{pic4}
\includegraphics[width=0.25\textwidth, height=0.25\textwidth]{pic5}
\includegraphics[width=0.25\textwidth, height=0.25\textwidth]{pic6}
\includegraphics[width=0.25\textwidth, height=0.25\textwidth]{pic7}
\includegraphics[width=0.25\textwidth, height=0.25\textwidth]{pic8}
\includegraphics[width=0.25\textwidth, height=0.25\textwidth]{pic9}
\caption{Sample plots for each individual digits from 0 to 9. }
 \end{figure}
 \item 
 
 
 \item 
 
 \item 
 
 
\end{enumerate} 
\end{document}

