\documentclass{article}

\begin{document} 

\title{Summary of Probability Seminar in 2015 Fall} 
\author{Jun Cheng} 
\maketitle


\begin{enumerate} 


\item September 1, 2015 \\
Topic:  Strong Hypercontractivity for Hypoelliptic Heat Kernels on Lie Groups. Speaker: Nate Eldredge 
This talk is about hypercontractivity. The speaker explained and discussed their recent work when replacing some setting with a complex nilpotent Lie Group. 

\item  September 15, 2015 \\
Topic: Oscillations of Quenched Slowdown Asymptotics for Ballistic One-Dimension Random Walk in A Random Environment. 
The speaker was talking about the one dimensional random walk in random environment with a positive speed $lim_{n\rightarrow\infty}\frac{X_n}{n}=\nu_a>0$. 

\item September 29, 2015 \\
Topic: Algebraic Error Estimates for the Stochastic Homogenization of Parabolic Equations. 
The speaker presented an overview of the stochastic homogenization of parabolic equations in spatio-temporal media and also the quenched error estimates. And also the difficulties that one may face when studying the spectral representation of NSA operators. 


\item October 20, 2015  \\
Topic:  Probabilistic Approach to Sato Grassmanian. Speaker: Jiro Akahori 
The speaker was talking about the Sato Grassmanian. 

\item October 27, 2015  \\
Topic: Stieltjes Transforms for Rapicdly or Algebraically Decaying Functions and Asymptotic Expansions, with Concrete Examples. Speaker: Anirban DasGupta. 
The speaker was talking about the application of Stieltjes transform, which involves dealing with the Stieltjes transform asymptotics only in the rapidly decaying case. 



\item November 10, 2015  \\
Topic: Spectral Representation of some Invariant Non-self-adjoint Semigroups and Hypocoercivity. Speaker: Pierre Patie. 
This talk is about the original methodology for developing the spectral representation of a class of non-self-adjoint invariant semigroups. 

\item December 1, 2015  \\
Topic: A Method of Rotation for Levy Multipliers. Speaker: Michael Perlmutter. 
The speaker talked about method of rotations to study the $L^p$ boundedness of Fourier multipliers. 

\item  December 8, 2015 \\
Topic: Choice and Order in Random Permutations.  Speaker: Nick Travers. 
There is a bias when selecting the lowest numbered of a variant of a simple procedure. They are trying to quantify this effect in terms of two natural measures of order: the number of inversions $I$ and the length of the longest increasing subsequence $L$.  


\end{enumerate}


\end{document}