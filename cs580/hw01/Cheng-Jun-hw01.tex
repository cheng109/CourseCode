\documentclass{article}
\usepackage[]{algorithm2e}
% Packages required to support encoding
\usepackage{ucs}
\usepackage[utf8x]{inputenc}
\usepackage{graphicx} 
% Packages required by code
\usepackage{enumitem}
% Packages always used
\usepackage{listings}
\usepackage{hyperref}
\usepackage{xspace}
\usepackage[usenames,dvipsnames]{color}
\hypersetup{colorlinks=true,urlcolor=blue}
\newcommand\floor[1]{\lfloor#1\rfloor}
\newcommand\ceil[1]{\lceil#1\rceil}

\usepackage[framed,numbered,autolinebreaks,useliterate] {mcode}
\usepackage{algpseudocode}

\usepackage{geometry}
\geometry{letterpaper,textwidth=350pt,textheight=680pt,tmargin=60pt,
            left=72pt,footskip=24pt,headsep=18pt,headheight=14pt}
\usepackage{amsmath}
\usepackage{amssymb}
\usepackage{textcase}
\usepackage{soul}

\newcommand{\mat}[1]{\boldsymbol{#1}}\renewcommand{\vec}[1]{\boldsymbol{\mathrm{#1}}}
\newcommand{\vecalt}[1]{\boldsymbol{#1}}

\newcommand{\conj}[1]{\overline{#1}}

\newcommand{\normof}[1]{\|#1\|}
\newcommand{\onormof}[2]{\|#1\|_{#2}}

\newcommand{\itr}[2]{#1^{(#2)}}
\newcommand{\itn}[1]{^{(#1)}}

\newcommand{\eps}{\varepsilon}
\newcommand{\kron}{\otimes}

\DeclareMathOperator{\diag}{diag}
\DeclareMathOperator{\trace}{trace}
\DeclareMathOperator{\tvec}{vec}

\newcommand{\prob}{\mathbb{P}}
\newcommand{\probof}[1]{\prob\left\{ #1 \right\}}

\newcommand{\pmat}[1]{\begin{pmatrix} #1 \end{pmatrix}}
\newcommand{\bmat}[1]{\begin{bmatrix} #1 \end{bmatrix}}
\newcommand{\spmat}[1]{\left(\begin{smallmatrix} #1 \end{smallmatrix}\right)}
\newcommand{\sbmat}[1]{\left[\begin{smallmatrix} #1 \end{smallmatrix}\right]}

\newcommand{\RR}{\mathbb{R}}
\newcommand{\CC}{\mathbb{C}}

\providecommand{\eye}{\mat{I}}
\providecommand{\mA}{\ensuremath{\mat{A}}}
\providecommand{\mB}{\ensuremath{\mat{B}}}
\providecommand{\mC}{\ensuremath{\mat{C}}}
\providecommand{\mD}{\ensuremath{\mat{D}}}
\providecommand{\mE}{\ensuremath{\mat{E}}}
\providecommand{\mF}{\ensuremath{\mat{F}}}
\providecommand{\mG}{\ensuremath{\mat{G}}}
\providecommand{\mH}{\ensuremath{\mat{H}}}
\providecommand{\mI}{\ensuremath{\mat{I}}}
\providecommand{\mJ}{\ensuremath{\mat{J}}}
\providecommand{\mK}{\ensuremath{\mat{K}}}
\providecommand{\mL}{\ensuremath{\mat{L}}}
\providecommand{\mM}{\ensuremath{\mat{M}}}
\providecommand{\mN}{\ensuremath{\mat{N}}}
\providecommand{\mO}{\ensuremath{\mat{O}}}
\providecommand{\mP}{\ensuremath{\mat{P}}}
\providecommand{\mQ}{\ensuremath{\mat{Q}}}
\providecommand{\mR}{\ensuremath{\mat{R}}}
\providecommand{\mS}{\ensuremath{\mat{S}}}
\providecommand{\mT}{\ensuremath{\mat{T}}}
\providecommand{\mU}{\ensuremath{\mat{U}}}
\providecommand{\mV}{\ensuremath{\mat{V}}}
\providecommand{\mW}{\ensuremath{\mat{W}}}
\providecommand{\mX}{\ensuremath{\mat{X}}}
\providecommand{\mY}{\ensuremath{\mat{Y}}}
\providecommand{\mZ}{\ensuremath{\mat{Z}}}
\providecommand{\mLambda}{\ensuremath{\mat{\Lambda}}}
\providecommand{\mPbar}{\bar{\mP}}

\providecommand{\ones}{\vec{e}}
\providecommand{\va}{\ensuremath{\vec{a}}}
\providecommand{\vb}{\ensuremath{\vec{b}}}
\providecommand{\vc}{\ensuremath{\vec{c}}}
\providecommand{\vd}{\ensuremath{\vec{d}}}
\providecommand{\ve}{\ensuremath{\vec{e}}}
\providecommand{\vf}{\ensuremath{\vec{f}}}
\providecommand{\vg}{\ensuremath{\vec{g}}}
\providecommand{\vh}{\ensuremath{\vec{h}}}
\providecommand{\vi}{\ensuremath{\vec{i}}}
\providecommand{\vj}{\ensuremath{\vec{j}}}
\providecommand{\vk}{\ensuremath{\vec{k}}}
\providecommand{\vl}{\ensuremath{\vec{l}}}
\providecommand{\vm}{\ensuremath{\vec{l}}}
\providecommand{\vn}{\ensuremath{\vec{n}}}
\providecommand{\vo}{\ensuremath{\vec{o}}}
\providecommand{\vp}{\ensuremath{\vec{p}}}
\providecommand{\vq}{\ensuremath{\vec{q}}}
\providecommand{\vr}{\ensuremath{\vec{r}}}
\providecommand{\vs}{\ensuremath{\vec{s}}}
\providecommand{\vt}{\ensuremath{\vec{t}}}
\providecommand{\vu}{\ensuremath{\vec{u}}}
\providecommand{\vv}{\ensuremath{\vec{v}}}
\providecommand{\vw}{\ensuremath{\vec{w}}}
\providecommand{\vx}{\ensuremath{\vec{x}}}
\providecommand{\vy}{\ensuremath{\vec{y}}}
\providecommand{\vz}{\ensuremath{\vec{z}}}
\providecommand{\vpi}{\ensuremath{\vecalt{\pi}}}

\sodef\allcapsspacing{\upshape}{0.15em}{0.65em}{0.6em}%

\makeatletter
\def\maketitle{%
\par
\hrule height 0.75pt\vspace{1ex}
\par\noindent
\begin{minipage}{0.5\textwidth}
\scshape
purdue university $\cdot$ CS 580 \\
Introduction to the Analysis of Algorithms
\end{minipage}
\begin{minipage}{0.5\textwidth}
\raggedleft
\MakeTextUppercase{\allcapsspacing{\@title}}\\[0.2ex]
\textit{\@author}\\[0.2ex]
\textit{\@date}
\end{minipage}
\par\vspace{1ex}
\hrule height 1pt
\vspace{2ex}
\par
}
\makeatother

\author{Jun Cheng}
\title{Lecture Notes}
% auto generate a title
\AtBeginDocument{\maketitle}


\title{Homework}



\begin{document} 

\hypertarget{}{}
\subsection*{{Problem 1: }}
\label{}
Using master theorem: \\
\begin{align*} 
a &=4 \\ 
b &= 4 \\ 
f(n) &= n\log{n} \\
\end{align*} 
Then \begin{align*} 
\log_b^a & = 1\\
f(n) &= \Omega(n^{\log_b^a-\epsilon})  
\end{align*} 
Therefore, \begin{align*} T(n) = \Theta(n\log{n}) \end{align*} 

\hypertarget{}{}
\subsection*{{Problem 2: }}
We guess $\Theta(n) $: 
\begin{align*} 
T(n) & = T(\floor{\frac{n}{7}}) + 5T(\floor{\frac{n}{6}}) + cn \\
	& = \left( \frac{1}{7} + \frac{5}{6}  + 1 \right) nc \\
	& = \frac{83}{42} nc 
\end{align*} 

\hypertarget{}{}
\subsection*{{Problem 3: }}


For the SoSoSplotchy numbers, we know that : 
\begin{align*} 
S(0) &= 1 \\
S(1) &=2 \\
S(n) &= 2S(n-1)+S(n-2) \\
\end{align*}
\begin{enumerate} [label=(\alph*)]
\item
To prove $S(n) = S(a+1)S(n-a-1) + S(a)S(n-2-a) $ \\
Basis:   $(a=0)$ and $n\geq2$ \\
\begin{align*} S(n)  &= S(a+1)S(n-a-1) + S(a)S(n-2-a) \\
	& = S(1)S(n-1) + S(0)S(n-2) \\
	& = 2S(n-1) + S(n-2)     \\
\end{align*} 
Induction step: $(a > 0)$  \\ 
If we know: 
\begin{align*} 
	S(n) & = S(a+1)S(n-a-1)+S(a)S(n-a-2)  \\
		& = S(a+1)[2S(n-a-2)+S(n-a-3)] + S(a)S(n-2-a) \\
		& = [2S(a+1) +S(a) ] S(n-a-2) + S(a+1) S(n-a-3)  \\ 
		& = S(a+2) S(n-a-2) + S(a+1) S(n-a-3)   \checkmark
\end{align*} 
Then for $a+1$, we will have 
\begin{align*} 
	S(n) & = S(a+2) S(n-a-2) + S(a+1) S(n-a-3) \\
		& = S(a+2)S(n-a-2) +S(a+1)S(n-a-3) \\
\end{align*}
	
% part (b) 	
\item 
Assuming the fact from part a, and let $n=2k$ and $a=k-1$, we will have: \\
\begin{align*} 
S(n) & = S(a+1)S(n-a-1) + S(a)S(n-2-a)\\
S(2k) &= S(k)S(k) + S(k-1)S(k-1) 
\end{align*} 
If we let $n=2k+1$ and $a+1=k+1$, then we will have: 
\begin{align*} 
S(2k+1) & = S(k+1)S(k) + S(k)S(k-1) \\
	& =[2S(k)+S(k-1)]S(k) + S(k)S(k-1)  \\
	& = 2S(k)S(k) + 2S(k-1)S(k)  \\
\end{align*} 
% part C
\item 
We know  $ S(2k+1)= 2S(k)S(k) + 2S(k-1)S(k) $ \\
Replace $k$ with $k-1$, then we will have: 
\begin{align*} 
S(2k-1) & =2S(k-1)S(k-1)+2S(k-1)S(k-2)  \\
	    & = 2S(k-1)S(k-1) +2S(k-1)[S(k) -2S(k-1)] \\
	    & = 2S(k-1)[S(k)-S(k-1) ]	 	
\end{align*} 
% part D
\item 
If $n$ is an odd number,  let $2k+1=n$  and $k=\frac{n-1}{2}$ , then 
\begin{align*} 
S(n) &= S(2k+1)  \\ 
	&= 2S(k)S(k) + 2S(k-1)S(k) \\
	& = 2S(\frac{n-1}{2})S(\frac{n-1}{2}) + 2S(\frac{n-3}{2})S(\frac{n-1}{2}) \\
S(n-1)  & = S(2k) \\
 	& = S(k)S(k) + S(k-1)S(k-1) \\
	& = S(\frac{n-1}{2})S(\frac{n-1}{2}) + S(\frac{n-3}{2})S(\frac{n-3}{2}) 
\end{align*}

If $n$ is an even number,  let $2k=n$  and $k=\frac{n}{2}$ , then 
\begin{align*} 
S(n) &= S(2k)  \\ 
	& = S(k)S(k) + S(k-1)S(k-1) \\
	& = S(\frac{n}{2})S(\frac{n}{2}) + S(\frac{n-2}{2}S(\frac{n-2}{2}) \\
S(n-1) &= S(2k-1)  \\ 
	&= 2S(k-1)S(k) - 2S(k-1)S(k-1) \\
	& = 2S(\frac{n-2}{2})S(\frac{n}{2}) - 2S(\frac{n-2}{2})S(\frac{n-2}{2}) \\
\end{align*}
Pseudocode:  \\
\begin{algorithm}[H]
Function $SoSoSplotchy (n)$: \\
\eIf{ $n=1$} 
	{return (2, 1)} 
{
\uIf {$n = odd$ }  
{\begin{align*}
	&S(\frac{n-1}{2}), S(\frac{n-3}{2}) = SoSoplotchy (\frac{n-1}{2})  \\
	&S(n) = 2S(\frac{n-1}{2})S(\frac{n-1}{2}) + 2S(\frac{n-3}{2})S(\frac{n-1}{2})\\
	&S(n-1) = S(\frac{n-1}{2})S(\frac{n-1}{2}) + S(\frac{n-3}{2})S(\frac{n-3}{2}) \\
\end{align*} }
\uIf {$n = even$} {
 \begin{align*}
	&S(\frac{n}{2}), S(\frac{n-2}{2}) = SoSoplotchy (\frac{n}{2})  \\
	&S(n) = S(\frac{n}{2})S(\frac{n}{2}) + S(\frac{n-2}{2}S(\frac{n-2}{2}) \\
	&S(n-1) = 2S(\frac{n-2}{2})S(\frac{n}{2}) - 2S(\frac{n-2}{2})S(\frac{n-2}{2}) \\
\end{align*}    }
}
return $S(n), S(n-1)$ 

\end{algorithm}
\item 

\end{enumerate}




\end{document}
