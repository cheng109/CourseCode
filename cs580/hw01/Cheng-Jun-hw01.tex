\documentclass{article}
\usepackage[]{algorithm2e}
% Packages required to support encoding
\usepackage{ucs}
\usepackage[utf8x]{inputenc}
\usepackage{graphicx} 
% Packages required by code
\usepackage{enumitem}
% Packages always used
\usepackage{listings}
\usepackage{hyperref}
\usepackage{xspace}
\usepackage[usenames,dvipsnames]{color}
\hypersetup{colorlinks=true,urlcolor=blue}
\newcommand\floor[1]{\lfloor#1\rfloor}
\newcommand\ceil[1]{\lceil#1\rceil}

\usepackage[framed,numbered,autolinebreaks,useliterate] {mcode}
\usepackage{algpseudocode}

\input{preamble.tex}

\title{Homework}



\begin{document} 

\hypertarget{}{}
\subsection*{{Problem 1: }}
\label{}
Using master theorem: \\
\begin{align*} 
a &=4 \\ 
b &= 4 \\ 
f(n) &= n\log{n} \\
\end{align*} 
Then \begin{align*} 
\log_b^a & = 1\\
f(n) &= \Omega(n^{\log_b^a-\epsilon})  
\end{align*} 
Therefore, \begin{align*} T(n) = \Theta(n\log{n}) \end{align*} 

\hypertarget{}{}
\subsection*{{Problem 2: }}
We guess $\Theta(n) $: 
\begin{align*} 
T(n) & = T(\floor{\frac{n}{7}}) + 5T(\floor{\frac{n}{6}}) + cn \\
	& = \left( \frac{1}{7} + \frac{5}{6}  + 1 \right) nc \\
	& = \frac{83}{42} nc 
\end{align*} 

\hypertarget{}{}
\subsection*{{Problem 3: }}


For the SoSoSplotchy numbers, we know that : 
\begin{align*} 
S(0) &= 1 \\
S(1) &=2 \\
S(n) &= 2S(n-1)+S(n-2) \\
\end{align*}
\begin{enumerate} [label=(\alph*)]
\item
To prove $S(n) = S(a+1)S(n-a-1) + S(a)S(n-2-a) $ \\
Basis:   $(a=0)$ and $n\geq2$ \\
\begin{align*} S(n)  &= S(a+1)S(n-a-1) + S(a)S(n-2-a) \\
	& = S(1)S(n-1) + S(0)S(n-2) \\
	& = 2S(n-1) + S(n-2)     \\
\end{align*} 
Induction step: $(a > 0)$  \\ 
If we know: 
\begin{align*} 
	S(n) & = S(a+1)S(n-a-1)+S(a)S(n-a-2)  \\
		& = S(a+1)[2S(n-a-2)+S(n-a-3)] + S(a)S(n-2-a) \\
		& = [2S(a+1) +S(a) ] S(n-a-2) + S(a+1) S(n-a-3)  \\ 
		& = S(a+2) S(n-a-2) + S(a+1) S(n-a-3)   \checkmark
\end{align*} 
Then for $a+1$, we will have 
\begin{align*} 
	S(n) & = S(a+2) S(n-a-2) + S(a+1) S(n-a-3) \\
		& = S(a+2)S(n-a-2) +S(a+1)S(n-a-3) \\
\end{align*}
	
% part (b) 	
\item 
Assuming the fact from part a, and let $n=2k$ and $a=k-1$, we will have: \\
\begin{align*} 
S(n) & = S(a+1)S(n-a-1) + S(a)S(n-2-a)\\
S(2k) &= S(k)S(k) + S(k-1)S(k-1) 
\end{align*} 
If we let $n=2k+1$ and $a+1=k+1$, then we will have: 
\begin{align*} 
S(2k+1) & = S(k+1)S(k) + S(k)S(k-1) \\
	& =[2S(k)+S(k-1)]S(k) + S(k)S(k-1)  \\
	& = 2S(k)S(k) + 2S(k-1)S(k)  \\
\end{align*} 
% part C
\item 
We know  $ S(2k+1)= 2S(k)S(k) + 2S(k-1)S(k) $ \\
Replace $k$ with $k-1$, then we will have: 
\begin{align*} 
S(2k-1) & =2S(k-1)S(k-1)+2S(k-1)S(k-2)  \\
	    & = 2S(k-1)S(k-1) +2S(k-1)[S(k) -2S(k-1)] \\
	    & = 2S(k-1)[S(k)-S(k-1) ]	 	
\end{align*} 
% part D
\item 
If $n$ is an odd number,  let $2k+1=n$  and $k=\frac{n-1}{2}$ , then 
\begin{align*} 
S(n) &= S(2k+1)  \\ 
	&= 2S(k)S(k) + 2S(k-1)S(k) \\
	& = 2S(\frac{n-1}{2})S(\frac{n-1}{2}) + 2S(\frac{n-3}{2})S(\frac{n-1}{2}) \\
S(n-1)  & = S(2k) \\
 	& = S(k)S(k) + S(k-1)S(k-1) \\
	& = S(\frac{n-1}{2})S(\frac{n-1}{2}) + S(\frac{n-3}{2})S(\frac{n-3}{2}) 
\end{align*}

If $n$ is an even number,  let $2k=n$  and $k=\frac{n}{2}$ , then 
\begin{align*} 
S(n) &= S(2k)  \\ 
	& = S(k)S(k) + S(k-1)S(k-1) \\
	& = S(\frac{n}{2})S(\frac{n}{2}) + S(\frac{n-2}{2}S(\frac{n-2}{2}) \\
S(n-1) &= S(2k-1)  \\ 
	&= 2S(k-1)S(k) - 2S(k-1)S(k-1) \\
	& = 2S(\frac{n-2}{2})S(\frac{n}{2}) - 2S(\frac{n-2}{2})S(\frac{n-2}{2}) \\
\end{align*}
Pseudocode:  \\
\begin{algorithm}[H]
Function $SoSoSplotchy (n)$: \\
\eIf{ $n=1$} 
	{return (2, 1)} 
{
\uIf {$n = odd$ }  
{\begin{align*}
	&S(\frac{n-1}{2}), S(\frac{n-3}{2}) = SoSoplotchy (\frac{n-1}{2})  \\
	&S(n) = 2S(\frac{n-1}{2})S(\frac{n-1}{2}) + 2S(\frac{n-3}{2})S(\frac{n-1}{2})\\
	&S(n-1) = S(\frac{n-1}{2})S(\frac{n-1}{2}) + S(\frac{n-3}{2})S(\frac{n-3}{2}) \\
\end{align*} }
\uIf {$n = even$} {
 \begin{align*}
	&S(\frac{n}{2}), S(\frac{n-2}{2}) = SoSoplotchy (\frac{n}{2})  \\
	&S(n) = S(\frac{n}{2})S(\frac{n}{2}) + S(\frac{n-2}{2}S(\frac{n-2}{2}) \\
	&S(n-1) = 2S(\frac{n-2}{2})S(\frac{n}{2}) - 2S(\frac{n-2}{2})S(\frac{n-2}{2}) \\
\end{align*}    }
}
return $S(n), S(n-1)$ 

\end{algorithm}
\item 

\end{enumerate}




\end{document}
