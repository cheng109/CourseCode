\documentclass{report} 
\usepackage{amsmath} 

\begin{document}
\title{Homework 5 report}
\author{Jun Cheng}
\maketitle

\section*{Problem 1} 
\subsection*{Part (a)} 
Buy-and-hold strategy: 

Rebalancing strategy: 
Parameters: 
\begin{itemize}
\item $S_0$: initial stock price.
\item $S_t$: stock price at time $t$.
\item $x_t$: number of stock shares at time $t$.
\item $C_t$: cash held at time $t$ in dollar. 
\end{itemize}
Use the Monte Carlo to simulate the stock price to get all $S_t$,  and then update all all $C_i $ and $x_i$: 
\begin{align*}
&C_{t+1}=\frac{x_tS_{t+1}}{2} \\
&x_{t+1}=\frac{C_1}{S_1}\\
\end{align*}
Monte Carlo simulations: 
\begin{itemize}
\item $ u=2$
\item $ d=0.5$
\item $p_u = p_d = 0.5$
\end{itemize} 
Running the attached code, we can get, 
\begin{align*}\begin{array}{ll}
E(U)=  & var(U) =\\
E(V)=   & var(V)=\\
\end{array} \end{align*}
To get $95\% $ confidence interval, we should $\delta = 0.05$, $z_{1-\delta/2}=1.96$, then the confidence interval
\begin{align*}
\begin{array}{ll}
\left[E(U)-1.96\frac{\sigma_u}{\sqrt{n}}, E(U)+1.96\frac{\sigma_u}{\sqrt{n}}\right]  :   [   ,  ]\\
\left[E(V)-1.96\frac{\sigma_v}{\sqrt{n}},  E(V)+1.96\frac{\sigma_v}{\sqrt{n}}\right]  :   [   ,  ]\\
\end{array}
\end{align*}

\subsection*{Part (b)}
Now we have new random variable $ T=V-U $.  Since $V$ and $U$ are independent, then we can have
\begin{align*}
& E(T) = E(V)-E(U) =  \\
& var(T) = var(T)+var(E) = \\
\end{align*}
then the confidence interval  
$$\left[E(T)-1.96\frac{\sigma_u}{\sqrt{n}}, E(T)+1.96\frac{\sigma_u}{\sqrt{n}}\right]  :   [   ,  ]\\ $$
 
\subsection*{Part(c)} 
If we use the same stream of random numbers then the confidence interval is
\begin{align*}
& E(V-U) = \\
& var(V-U) = \\
& Confidence interval :  [ , ] \\
\end{align*}
This confidence is wider than that of Part(b). 

\subsection*{Part(d)} 
If we use the same stream of random numbers then the confidence interval is
\begin{align*}
& E(log_{10}V-log_{10}U) = \\
& var(log_{10}V-log_{10}U) = \\
& Confidence interval :  [ , ] \\
\end{align*}
Compare with Part(c) using utility functions gives a better comparison of investment alternatives.
 
\section*{Problem 2}


\section*{Problem 3}

\section*{Problem 4} 


\end{document}