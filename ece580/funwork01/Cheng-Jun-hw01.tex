\documentclass{article}

% Packages required to support encoding
\usepackage{ucs}
\usepackage[utf8x]{inputenc}
\usepackage{graphicx} 
% Packages required by code

% Packages always used
\usepackage{listings}
\usepackage{hyperref}
\usepackage{xspace}
\usepackage[usenames,dvipsnames]{color}
\hypersetup{colorlinks=true,urlcolor=blue}


\usepackage[framed,numbered,autolinebreaks,useliterate] {mcode}


\input{preamble.tex}

\title{Homework}



\begin{document} 

\hypertarget{}{}
\subsection*{{Problem 1: }}
(Contraposition):  Suppose $x, y \in \mathbf{R}$, If $ y^3+yx^2\leq x^3+xy^2$, then $y\leq x$.  \\
To prove by contraposition, assume $y > x $, and $y$ and $x$ can not be $0$ at the same time. 
Then\begin{align*}
x^2+y^2 &> 0  \\
y(y^2+x^2) &> x(x^2+y^2) \\
 y^3+yx^2 &> x^3+xy^2 \\
\end{align*}
So $y^3+yx^2\leq x^3+xy^2$ is not true. Therefore the statement is true. 


\subsection*{{Problem 2: }}




\subsection*{{Problem 3: }}


\hypertarget{}{}
\subsection*{{Problem 4: }}
\label{}
\begin{enumerate} 
\item 
\begin{align*} 
\mA &= \bmat{1 & 1 & 2 & 1 \\ 1 & 2 & 4 & -2 }  \\ 
\vb &= \bmat{1 \\ 0 }  \\
rank(\mA) &= 2 \\
rank(\mA, \vb) &= 2 \\
\end{align*} 
Based on Theorem 2.1, $rank(\mA) = rank(\mA,\vb) $, then the system has a solution. 
\begin{align*}
\bmat{1 & 1\\ 1 & 2} \bmat{x_1 \\ x_2} &= \bmat{1-2d_3-d_4 \\ -4x_3-2x_4}  \\
\bmat{x_1\\ x_2} &= \bmat{2 & -1 \\ -1 & 1} \bmat{1-2d_3-d_4 \\ -4x_3-2x_4} \\
& = \bmat{2 \\ -1-2d_3}
\end{align*}


\item 
\begin{align*}
\mA& =\bmat{2 & 1 & 2 & 1\\ 6 & 3 & 6 & 3} \\
\vb &= \bmat{0 \\ 1 } \\
rank(\mA) &= 1\\
rank(\mA, \vb)& = 2
\end{align*} 
Based on Theorem 2.1, $rank(\mA) \neq rank(\mA, \vb)$, then the system has no solutions. 
\end{enumerate}



\hypertarget{}{}
\subsection*{{Problem 5: }}
\label{}
\begin{align*}
\mA &= \bmat{c & 0 & a \\ 0 & c & b \\ b & a & 0 } \\
\vb &= \bmat{b \\ a \\ c } \\
\mA | \vb &= \bmat {c & 0 & a \\ 0 & c & b \\ 0 & 0 & -\frac{2ab}{c} } | \bmat{ b \\ a \\  \frac{c^2 - a^2 - b^2}{c}}
\end{align*}
The one unique solution for this linear system is:  \begin{align*} a & \neq 0 \\ b & \neq 0 \\ c & \neq 0 \\ \end{align*} 
The solution is : 
\begin{align*} 
x_1 & = \frac{b^2+c^2-a^2}{2bc} \\
x_2 & = \frac{a^2+c^2-b^2}{2ac} \\
x_3 & = \frac{a^2+b^2-c^2}{2bc} \\ 
\end{align*} 


\hypertarget{}{}
\subsection*{{Problem 6: }}
\label{}
\begin{align*} 
&\left[   \begin{array}{cc} 
\cos x & \sin x \\ 
-\sin x & \cos x \\ 
\end{array}  \right] \left[ \begin{array}{cc}
\cos y & \sin y \\ 
-\sin y & \cos y \\ 
\end{array}  \right] \\&= \left[   \begin{array}{cc} 
\cos x\cos y -\sin x\sin y & \cos x\sin y + \sin x\cos y \\ 
-\sin x\cos y-\sin y\cos x &\cos x\cos y -\sin x\sin y\\ 
\end{array}  \right] \\&= \left[   \begin{array}{cc} 
\cos (x+y) & \sin (x+y) \\ 
-\sin (x+y) & \cos (x+y) \\ 
\end{array}  \right]
\end{align*}
Then \begin{align*} 
&\left[   \begin{array}{cc} 
\cos x & \sin x \\ 
-\sin x & \cos x \\ 
\end{array}  \right]^2 = \left[   \begin{array}{cc} 
\cos 2x & \sin 2x \\ 
-\sin 2x & \cos 2x \\ 
\end{array}  \right]
\end{align*}
\begin{align*} 
&\left[   \begin{array}{cc} 
\cos x & \sin x \\ 
-\sin x & \cos x \\ 
\end{array}  \right]^{57} = \left[   \begin{array}{cc} 
\cos 57x & \sin 57x \\ 
-\sin 57x & \cos 57x \\ 
\end{array}  \right]
\end{align*}

\hypertarget{}{}
\subsection*{{Problem 7: }}
\label{}
\begin{align*}
\mA&=\bmat{1 & 2 \\ 3 & 0} \\
f_1(x) &= x_2 -2x +5  \\
f_2(x) &= 7x +5 \\ 
\end{align*} 
Then \begin{align*} 
f_1(\mA) & =  \bmat{1 & 2 \\ 3 & 0}\bmat{1 & 2 \\ 3 & 0} - 2\bmat{1 & 2 \\ 3 & 0}+ \bmat{5 & 0 \\ 0 & 5} \\
& = \bmat{10 & -2 \\ -3 & 11} \\ 
f_2(\mA) & = 7\bmat{1 & 2 \\ 3 & 0} + \bmat{5 & 0 \\ 0 & 5} \\
& = \bmat{12 & 14 \\ 21 & 5} \\
5f_1(\mA)-3f_2(\mA) & = \bmat{14 & -62 \\ 78 & 40}
\end{align*}

\hypertarget{}{}
\subsection*{{Problem 8: }}
\label{}
\begin{enumerate}
\item
$f(x_1, x_2) = x_1^2 + x_2^2 + 4x_1x_2 +\frac{2}{3}x_2^3-2x_2 + 7$ \\
\begin{align*}
\frac{\partial f}{\partial x_1} &= 2x_1+4x_2 = 0\\
\frac{\partial f}{\partial x_2} &= 2x_2 +4x_1 + 2x_2^2 - 2 = 0 \\
 x_1 &= -6.6 \\ x_2 & = 3.3 \\
 or \\
 x_1 & = 0.6 \\ x_2 & = -0.3 \\
\end{align*}
So there are two extremum points $(-6.6, 3.3) $ and $(0.6, -0.3) $ 
\item 
\begin{align*}
\frac{\partial^2 f}{\partial x_1^2} &= 2 >0 \\
\frac{\partial^2 f}{\partial x_2^2} &= 2+4x_2>0
\end{align*} 
 Then we need $x_2>-0.5$, and both points satisfies these conditions. Therefore both points are local minimizers. 
\end{enumerate}

\hypertarget{}{}
\subsection*{{Problem 9: }}
\label{}
\begin{enumerate}
\item 
$f(x_1, x_2,x_3, x_4) = 7x_1^2+x_3^2-2x_1x_3+x_1x_4$ \\
\begin{align*} 
f &=\frac{1}{2}\vx^T\mQ\vx \\
& =\frac{1}{2} \bmat{x_1 &  x_2 & x_3 & x_4 } \bmat{ 14 & 0 & -2 & 1 \\ 0 & 0 & 0 & 0 \\ -2 & 0 & 2 & 0 \\ 1 & 0 & 0 & 0  } \bmat{x_1 \\  x_2 \\  x_3 \\ x_4}
\end{align*}
\item 
$f(x_1, x_2,x_3) = x_2^2-3x_1x_2$\\
\begin{align*} 
f &=\frac{1}{2}\vx^T\mQ\vx \\
& =\frac{1}{2} \bmat{x_1 &  x_2 & x_3 } \bmat{ 0 & -3 & 0 \\ -3 & 2 & 0 \\ 0 & 0 & 0  } \bmat{x_1 \\  x_2 \\  x_3 }
\end{align*}
\item 
$f(x_1, x_2,x_3) = 2x_1^2-5x_2^2 +2x_1x_2 $
\begin{align*} 
f &=\frac{1}{2}\vx^T\mQ\vx \\
& =\frac{1}{2} \bmat{x_1 &  x_2 & x_3 } \bmat{ 4 & 2 & 0 \\ 2 & -5 & 0 \\ 0 & 0 & 0  } \bmat{x_1 \\  x_2 \\  x_3 }
\end{align*}
\end{enumerate}




\end{document}
