\documentclass{article}

% Packages required to support encoding
\usepackage{ucs}
\usepackage[utf8x]{inputenc}
\usepackage{graphicx} 
% Packages required by code
\usepackage{enumitem}
% Packages always used
\usepackage{listings}
\usepackage{hyperref}
\usepackage{xspace}
\usepackage[usenames,dvipsnames]{color}
\hypersetup{colorlinks=true,urlcolor=blue}

\newcommand*\rfrac[2]{{}^{#1}\!/_{#2}}
\usepackage[framed,numbered,autolinebreaks,useliterate] {mcode}


\usepackage{geometry}
\geometry{letterpaper,textwidth=350pt,textheight=680pt,tmargin=60pt,
            left=72pt,footskip=24pt,headsep=18pt,headheight=14pt}
\usepackage{amsmath}
\usepackage{amssymb}
\usepackage{textcase}
\usepackage{soul}

\newcommand{\mat}[1]{\boldsymbol{#1}}\renewcommand{\vec}[1]{\boldsymbol{\mathrm{#1}}}
\newcommand{\vecalt}[1]{\boldsymbol{#1}}

\newcommand{\conj}[1]{\overline{#1}}

\newcommand{\normof}[1]{\|#1\|}
\newcommand{\onormof}[2]{\|#1\|_{#2}}

\newcommand{\itr}[2]{#1^{(#2)}}
\newcommand{\itn}[1]{^{(#1)}}

\newcommand{\eps}{\varepsilon}
\newcommand{\kron}{\otimes}

\DeclareMathOperator{\diag}{diag}
\DeclareMathOperator{\trace}{trace}
\DeclareMathOperator{\tvec}{vec}

\newcommand{\prob}{\mathbb{P}}
\newcommand{\probof}[1]{\prob\left\{ #1 \right\}}

\newcommand{\pmat}[1]{\begin{pmatrix} #1 \end{pmatrix}}
\newcommand{\bmat}[1]{\begin{bmatrix} #1 \end{bmatrix}}
\newcommand{\spmat}[1]{\left(\begin{smallmatrix} #1 \end{smallmatrix}\right)}
\newcommand{\sbmat}[1]{\left[\begin{smallmatrix} #1 \end{smallmatrix}\right]}

\newcommand{\RR}{\mathbb{R}}
\newcommand{\CC}{\mathbb{C}}

\providecommand{\eye}{\mat{I}}
\providecommand{\mA}{\ensuremath{\mat{A}}}
\providecommand{\mB}{\ensuremath{\mat{B}}}
\providecommand{\mC}{\ensuremath{\mat{C}}}
\providecommand{\mD}{\ensuremath{\mat{D}}}
\providecommand{\mE}{\ensuremath{\mat{E}}}
\providecommand{\mF}{\ensuremath{\mat{F}}}
\providecommand{\mG}{\ensuremath{\mat{G}}}
\providecommand{\mH}{\ensuremath{\mat{H}}}
\providecommand{\mI}{\ensuremath{\mat{I}}}
\providecommand{\mJ}{\ensuremath{\mat{J}}}
\providecommand{\mK}{\ensuremath{\mat{K}}}
\providecommand{\mL}{\ensuremath{\mat{L}}}
\providecommand{\mM}{\ensuremath{\mat{M}}}
\providecommand{\mN}{\ensuremath{\mat{N}}}
\providecommand{\mO}{\ensuremath{\mat{O}}}
\providecommand{\mP}{\ensuremath{\mat{P}}}
\providecommand{\mQ}{\ensuremath{\mat{Q}}}
\providecommand{\mR}{\ensuremath{\mat{R}}}
\providecommand{\mS}{\ensuremath{\mat{S}}}
\providecommand{\mT}{\ensuremath{\mat{T}}}
\providecommand{\mU}{\ensuremath{\mat{U}}}
\providecommand{\mV}{\ensuremath{\mat{V}}}
\providecommand{\mW}{\ensuremath{\mat{W}}}
\providecommand{\mX}{\ensuremath{\mat{X}}}
\providecommand{\mY}{\ensuremath{\mat{Y}}}
\providecommand{\mZ}{\ensuremath{\mat{Z}}}
\providecommand{\mLambda}{\ensuremath{\mat{\Lambda}}}
\providecommand{\mPbar}{\bar{\mP}}

\providecommand{\ones}{\vec{e}}
\providecommand{\va}{\ensuremath{\vec{a}}}
\providecommand{\vb}{\ensuremath{\vec{b}}}
\providecommand{\vc}{\ensuremath{\vec{c}}}
\providecommand{\vd}{\ensuremath{\vec{d}}}
\providecommand{\ve}{\ensuremath{\vec{e}}}
\providecommand{\vf}{\ensuremath{\vec{f}}}
\providecommand{\vg}{\ensuremath{\vec{g}}}
\providecommand{\vh}{\ensuremath{\vec{h}}}
\providecommand{\vi}{\ensuremath{\vec{i}}}
\providecommand{\vj}{\ensuremath{\vec{j}}}
\providecommand{\vk}{\ensuremath{\vec{k}}}
\providecommand{\vl}{\ensuremath{\vec{l}}}
\providecommand{\vm}{\ensuremath{\vec{l}}}
\providecommand{\vn}{\ensuremath{\vec{n}}}
\providecommand{\vo}{\ensuremath{\vec{o}}}
\providecommand{\vp}{\ensuremath{\vec{p}}}
\providecommand{\vq}{\ensuremath{\vec{q}}}
\providecommand{\vr}{\ensuremath{\vec{r}}}
\providecommand{\vs}{\ensuremath{\vec{s}}}
\providecommand{\vt}{\ensuremath{\vec{t}}}
\providecommand{\vu}{\ensuremath{\vec{u}}}
\providecommand{\vv}{\ensuremath{\vec{v}}}
\providecommand{\vw}{\ensuremath{\vec{w}}}
\providecommand{\vx}{\ensuremath{\vec{x}}}
\providecommand{\vy}{\ensuremath{\vec{y}}}
\providecommand{\vz}{\ensuremath{\vec{z}}}
\providecommand{\vpi}{\ensuremath{\vecalt{\pi}}}

\sodef\allcapsspacing{\upshape}{0.15em}{0.65em}{0.6em}%

\makeatletter
\def\maketitle{%
\par
\hrule height 0.75pt\vspace{1ex}
\par\noindent
\begin{minipage}{0.5\textwidth}
\scshape
purdue university $\cdot$ CS 580 \\
Introduction to the Analysis of Algorithms
\end{minipage}
\begin{minipage}{0.5\textwidth}
\raggedleft
\MakeTextUppercase{\allcapsspacing{\@title}}\\[0.2ex]
\textit{\@author}\\[0.2ex]
\textit{\@date}
\end{minipage}
\par\vspace{1ex}
\hrule height 1pt
\vspace{2ex}
\par
}
\makeatother

\author{Jun Cheng}
\title{Lecture Notes}
% auto generate a title
\AtBeginDocument{\maketitle}


\title{Homework}



\begin{document} 

\hypertarget{}{}
\subsection*{{Problem 1: Exercise 16.12}}
\label{}
Solve the following linear programs using simplex method:\\ 
(a) Maximize $-4x_1 -3x_2$ subject to 
\begin{align*}
5x_1 + x_2 &\geq 11 \\
-2x_1 -x_2  & \leq -8  \\
x_1 + 2x_2 &\geq 7 \\
x_1, x_2 & \geq 0 
\end{align*}

Introduce slack variables $x_3, x_4, x_5$, we will have: \\ 
\begin{align*} 
5x_1 + x_2 -x_3 &= 11 \\
-2x_1 -x_2 +x_4 &= -8 \\
x_1 + 2x_2 -x_5 &= 7 \\
x_1, x_2, x_3, x_4, x_5 & \geq 0 \\
\end{align*} 
We are trying to minimize $4x_1+3x_2$ \\ 
\begin{align*} 
C = [ 4, 3, 0, 0, 0] \\
\end{align*} 

Then \\
\begin{align*} 
\bmat{5 & 1 & -1 & 0 & 0  & 11 \\ 2 & 1 & 0 & -1 & 0 & 8 \\ 1 & 2 &  0 & 0 & -1 & 7 } 
\end{align*} 
Now we are using Two-Phase simplex method to solve this problem. \\
\begin{align*}
\bmat{5 & 1 & -1 & 0 & 0 & 1 & 0 & 0 & 11 \\ 2 & 1 & 0 & -1 & 0 & 0 & 1 & 0 & 8 \\ 1 & 2 & 0 & 0 & -1 & 0 & 0 & 1 & 7\\ 0 & 0 & 0 & 0 & 0 & 1 & 1 & 1 & 0 }
 \end{align*}
We must update the last row : \\
\begin{align*}
\bmat{5 & 1 & -1 & 0 & 0 & 1 & 0 & 0 & 11 \\ 2 & 1 & 0 & -1 & 0 & 0 & 1 & 0 & 8 \\ 1 & 2 & 0 & 0 & -1 & 0 & 0 & 1 & 7\\ -8 & -4 & 1 & 1 & 1 & 0 & 0 & 0 & -26 }
\end{align*}
Take $\alpha_{11} $ as pivot: 
\begin{align*}
\bmat{1 & \rfrac{1}{5} & -\rfrac{1}{5} & 0 & 0 &  \rfrac{1}{5} & 0 & 0 & \rfrac{11}{5} \\
0 & \rfrac{3}{5} & \rfrac{2}{5} & -1 & 0 & \rfrac{2}{5} & 1 & 0 & \rfrac{18}{5} \\
0 & \rfrac{9}{5} & \rfrac{1}{5} & 0 & -1 & -\rfrac{1}{5} & 0 & 1 & \rfrac{24}{5} \\
0 & -\rfrac{12}{5} & -\rfrac{3}{5} &  1 & 1 & \rfrac{8}{5} & 0 & 0 & \rfrac{42}{5} \\} 
\end{align*}
Take $\alpha_{31} $ as pivot: 
\begin{align*} 
\bmat{1 & 0 & \rfrac{2}{9} & 0 & \rfrac{1}{9} & \rfrac{2}{9} & 0 & \rfrac{1}{9} & \rfrac{5}{3} \\
0 & 0 & \rfrac{1}{3} & -1 & \rfrac{1}{3} & -\rfrac{1}{3} &  1 & -\rfrac{1}{3} & 2 \\
0 & 1 & \rfrac{1}{9} & 0 & -\rfrac{5}{9} & -\rfrac{1}{9} & 0 & \rfrac{5}{9} & \rfrac{8}{3} \\
0 & 0 & \rfrac{1}{3} & 1 & -\rfrac{1}{3} & \rfrac{4}{3} & 0 & \rfrac{4}{3} & -2} 
\end{align*}
Take $\alpha_{23}$ as pivot: 
\begin{align*} 
\bmat{1 & 0 & 0 & \rfrac{2}{3} & \rfrac{1}{3} & 0 & \rfrac{2}{3} & -\rfrac{1}{3} & 3 \\
0 & 0 & 1 & -3 & 1 & -1 & 3 & -1 & 6 \\
0 & 1 & 0 & \rfrac{1}{3} & -\rfrac{2}{3} & 0 & -3 & \rfrac{2}{3} & 2 \\
0 & 0 & 0 & 0 & 0 & 1 & 1 & 1 & 0} 
\end{align*} 
Now we can remove column 6-8. 
\begin{align*}
\bmat{1& 0 & 0 & -\rfrac{2}{3} & \rfrac{1}{3} & 3 \\
0 & 0 & 1 & -3 & 1 & 6 \\
0 & 1 & 0 & \rfrac{1}{3} & -\rfrac{2}{3} & 2 \\
4 & 3 & 0 & 0 & 0 & 0 \\}
\end{align*} 
Updating the last row, we can have: \\
\begin{align*} 
\bmat{1& 0 & 0 & -\rfrac{2}{3} & \rfrac{1}{3} & 3 \\
0 & 0 & 1 & -3 & 1 & 6 \\
0 & 1 & 0 & \rfrac{1}{3} & -\rfrac{2}{3} & 2 \\
0 & 0 & 0 & \rfrac{5}{3} & 0 & -18 \\}
\end{align*}
All the reduced cost coefficients are nonnegative, hence the optimal solution is \begin{align*} 
x = \bmat{3 \\ 2\\ 6\\ 0\\ 0}  \end{align*} 
and the optimal value is $18$. 

\hypertarget{}{}
\subsection*{{Problem 2: Exercise 16.12}}
\label{}
The dual problem:  Maximize $11\lambda_1 + 8\lambda_2 + 7 \lambda_3 $ subject to:
\begin{align*} 
5\lambda_1 + 2 \lambda_2 + \lambda_3 &\leq 4 \\
\lambda_1 +  \lambda_2 + 2\lambda_3 &\leq 3 \\
\lambda_1 ,  \lambda_2 , \lambda_3 & \geq 0 \\
\end{align*}
If we introduce slack variables $\lambda_4$ and $\lambda_5$, we will have :\begin{align*} 
5\lambda_1 + 2 \lambda_2 + \lambda_3 + \lambda_4  &= 4 \\
\lambda_1 +  \lambda_2 + 2\lambda_3 + \lambda_5 &= 3 \\
\lambda_1 ,  \lambda_2 , \lambda_3, \lambda_4 , \lambda_5 & \geq 0 \\
\end{align*}
Then \begin{align*} 
\bmat{5 & 2 & 1 & 1 & 0 & 4 \\ 1 & 1 & 2 & 0 & 1 & 3 \\ -11 & -8 & -7 & 0 & 0 & 0 }
\end{align*}
Take $\alpha_{11}$ as pivot to get : \begin{align*}
\bmat{1 & \rfrac{2}{5} & \rfrac{1}{5} & \rfrac{1}{5} & 0 & \rfrac{4}{5} \\ 0 & \rfrac{3}{5} & \rfrac{9}{5} & -\rfrac{1}{5} & 1 & \rfrac{11}{5} \\
0 & -\rfrac{18}{5} & -\rfrac{24}{5} & \rfrac{11}{5} & 0 & \rfrac{44}{5} }
\end{align*}
Take $\alpha_{23}$ as pivot to get : \begin{align*}
\bmat{1 & \rfrac{1}{3} & 0 & \rfrac{2}{9} & -\rfrac{1}{9} &  \rfrac{5}{9} \\ 0 & \rfrac{1}{3} &1 & -\rfrac{1}{9} & \rfrac{5}{9} & \rfrac{11}{9} \\
0 & -2 & 0 & \rfrac{5}{3} & \rfrac{8}{3} & \rfrac{44}{3} }
\end{align*} 
Take $\alpha_{12}$ as pivot to get : \begin{align*}
\bmat{3 & 1 & 0 & \rfrac{2}{3} & -\rfrac{1}{3} & \rfrac{5}{3} \\
-1 & 0 & 1 & -\rfrac{1}{3} & \rfrac{2}{3} & \rfrac{2}{3}  \\
6 & 0 & 0 & 3 & 2 & 18 }
\end{align*}

All the reduced cost coefficients are nonnegative, hence the optimal solution is \begin{align*} 
\lambda = \bmat{0 \\ \rfrac{5}{3} \\ \rfrac{2}{3} \\ 0\\ 0}  \end{align*} 
and the optimal value is $18$. 








\hypertarget{}{}
\subsection*{{Problem 3: Exercise 20.2 b. }}
\label{}
 Find  local extremizers for the following optimization problem: \\
 Minimize:  $ 4x_1 + x_2^2 $   \\
 subject to :  $ x_1^2 + x_2^2  = 9 $\\
 
\begin{align*} 
f(x_1,x_2)  & = 4x_1 + x_2^2 \\
h(x_1, x_2) & = x_1^2 + x_2^2 - 9 \\ 
\triangledown f  & = \bmat{ 4 &  2x_2}  \\
\triangledown h & = \bmat{2x_1 & 2x_2 } \\
\end{align*} 
By the Lagrange condition:  \begin{align*} 
4-\lambda\times 2 x_1 & = 0 \\
2x_2 -\lambda\times 2 x_2 & = 0 \\
x_1^2 + x_2^2 - 9 & = 0 \\ 
\end{align*} 
If $\lambda = -1 $, we have: 
\begin{align*} 
\vx &= \bmat{2 & \sqrt{5} } \\
\vx &= \bmat{2 & -\sqrt{5} } 
\end{align*} 
If $\lambda = \frac{2}{3}$, we have:  $\vx = \bmat{-3 & 0 }  $ \\
If $\lambda = -\frac{2}{3}$, we have:  $\vx = \bmat{3 & 0 }  $ \\
Since we want maximizers only, we keep $\lambda = -1$. Now we want to show that 
\begin{align*}
\vx &= \bmat{2 & \sqrt{5} }\\
 \vx &= \bmat{2 & -\sqrt{5} } 
 \end{align*}  are indeed the maximizers. 
The Hessian matrix is :
\begin{align*} 
H(x_1, x_2) &= \bmat{0 & 0 \\ 0 & 2 } + \lambda\bmat{2 & 0 \\ 0 & 2 } \\
& = \bmat{-2 & 0 \\ 0 & 0 }
 \end{align*}
 To find the tangent space we have : 
 \begin{align*} 
 \bmat{x_1 & x_2 } \bmat{y_1 \\ y_2 } 
 =  \bmat{2 & \pm\sqrt{5} } \bmat{y_1 \\ y_2 }  = 0 
 \end{align*}
 So the tangent space is $\bmat{\frac{\sqrt{5}}{2}a, & a } $ and $\bmat{-\frac{\sqrt{5}}{2}a, & a } $ 
 Then, \begin{align*}
 \bmat{\frac{\sqrt{5}}{2}a, &a} \bmat{-2 & 0 \\ 0 & 0 } \bmat{\frac{\sqrt{5}}{2}a \\a} &= -\frac{5}{2}a^2 < 0 \\
 \bmat{\frac{-\sqrt{5}}{2}a, &a} \bmat{-2 & 0 \\ 0 & 0 } \bmat{\frac{-\sqrt{5}}{2}a \\a} &= -\frac{5}{2}a^2 < 0 
 \end{align*} 
 Therefore, \begin{align*}
\vx &= \bmat{2 & \sqrt{5} }\\
 \vx &= \bmat{2 & -\sqrt{5} } 
 \end{align*}  are indeed the maximizers.
 
 
  
\hypertarget{}{}
\subsection*{{Problem 4: Exercise 20.9}}
\label{}
Find all maximizers of the function: $f(x_1, x_2) = \frac{18x_1^2-8x_1x_2 + 12x_2^2}{2x_1^2 + 2x_2^2 } $ 
In this problem we have 
\begin{align*} 
\mP & = \bmat{ 2 & 0 \\ 0 & 2 }  \\
\mQ & = \bmat{18 & -4 \\ -4 & 12} \\
\end{align*} 

Then we can have : \\ 
\begin{align*} 
\mP^{-1}\mQ = \bmat{9 & -2 \\ -2 & 6 } \\
\end{align*} 
The eigenvalues are $\lambda = 10 $ and $\lambda = 5$, 
Since we are looking for maximizer, we keep $\lambda=10$ only. 
Now we will show that the $\bmat{-\frac{2}{\sqrt{10}},  \frac{1}{\sqrt{10} }}$  and  $\bmat{\frac{2}{\sqrt{10}},  -\frac{1}{\sqrt{10} }}$and is indeed the maximizer.\\
The Hessian is \begin{align*} 
H(x_1, x_2) = 2\mQ - 2\lambda\mP  = \bmat{-8 & -8 \\ -8 & -16 } 
\end{align*}
 To find the tangent space we have:  \begin{align*} 
 \bmat{4x^*_1, 4x^*_2} \bmat{y_1 \\ y_2} &=  0 \\
\bmat{y_1 \\ y_2 } &= \bmat{a \\ 2a}  
\end{align*}
Then \begin{align*} 
\bmat{a & 2a} \bmat{-8 & -8 \\ -8 & -16 }  \bmat{y_1 \\ y_2}  \bmat{a \\ 2a} = -104a^2 <0 
\end{align*}
Therefore, \begin{align*} 
 \vx & =  \bmat{-\frac{2}{\sqrt{10}},  \frac{1}{\sqrt{10} }} \\
 \vx & =  \bmat{\frac{2}{\sqrt{10}},  -\frac{1}{\sqrt{10} }} 
\end{align*}
are indeed the maximizers. 

\hypertarget{}{}
\subsection*{{Problem 5: Exercise 21.2}}
\label{}
Find local extremizers for : \\
a. $x_1^2 + x_2^2 -2x_1 - 10x_2 + 26 $,  subject to $\rfrac{1}{5} x_2 -x_1^2 \leq 0, 5x_1+\rfrac{1}{2}x_2\leq5$ \\
b. $x_1^2 +x_2^2 $, subject to $x_1 \geq0, x_2\geq0, x_1+x_2\geq 5$ \\
c. $x_1^2 +6x_1x_2 -4x_1 -2x_2 $,  subject to $x_1^2 +2x_2\leq1, 2x_1-2x_2 \leq 1$ 

\begin{enumerate} [label=(\alph*)] 

\item 
\begin{align*} 
f(\vx) & = x_1^2 + x_2^2 -x_1 - 10x_2 + 26 \\
g_1 (\vx)  &=  \rfrac{1}{5} x_2 -x_1^2  \leq 0 \\
g_2(\vx) & =  5x_1+\rfrac{1}{2}x_2 - 5 \leq 0 \\
\triangledown f(\vx) &= \bmat{2x_1 - 2,  & 2x_2-10} \\
\triangledown g_1(\vx) &= \bmat{-2x_1, & \rfrac{1}{5} } \\
\triangledown g_2(\vx) &= \bmat{ 5, & \rfrac{1}{2}} 
\end{align*} 
Write the KKT condition: 
\begin{align*} 
(2x_1-2) -2 \mu_1 x_1  + 5\mu_2 &= 0 \\
(2x_2-10) +\frac{1}{5}\mu_1 + \frac{1}{2} \mu_2 & = 0 \\
\mu_1 (\frac{1}{5}x_2 -x_1^2 ) + \mu_2(5x_1-\frac{1}{2} - 5) & = 0 \\
\end{align*} 
If  $\mu_1 = 0  $ and $\mu_2  = 0 $, we have \begin{align*} \vx = \bmat{1, & 5} \end{align*} which does not satisfy $g_2(x) \leq 0 $, so no feasible solutions.  \\
If  $\mu_1 \neq 0  $ and $\mu_2  = 0 $, we have only one feasible solution:  \begin{align*} \vx = \bmat{-1.02, & 5.2} \end{align*}  \\
If  $\mu_1 = 0  $ and $\mu_2  \neq 0 $, we do not have only one feasible solution. \\
If  $\mu_1 \neq 0  $ and $\mu_2 \neq 0 $, we have only one feasible solution:  \begin{align*} \vx = \bmat{-1+\sqrt{2},  & 3-2\sqrt{2}} \end{align*}  \\

\item 
\begin{align*} 
f(\vx) & = x_1^2 + x_2 ^ 2 \\
g_1(\vx) & = x_1 \geq 0 \\
g_2(\vx) & = x_2 \geq 0 \\
g_3(\vx) & = x_1 +x_2 -5 \geq 0 \\
\triangledown f(\vx) &= \bmat{2x_1, & 2x_2} \\
\triangledown g_1(\vx) &= \bmat{1, & 0 } \\
\triangledown g_2(\vx) &= \bmat{0, & 1}  \\
\triangledown g_3(\vx) &= \bmat{1, & 1}  \\
\end{align*}
Then we have \begin{align*} 
2x_1 + \mu_1 +\mu_3 & = 0 \\
2x_2 + \mu_2 +\mu_3 & = 0 \\
x_1 & \geq 0 \\
x_2 & \geq 0 \\
\mu_1 x_1 + \mu_2x_2 + \mu_3 (x_1 + x_2 -5)  & = 0 \\
\end{align*} 
If $\mu_1 = \mu_2 =  0$ and $\mu_3\neq 0 $  then $x_1= x_2 = \frac{5}{2} $\\
After trying different combination of $\mu_1$, $\mu_2$ and $\mu_3$ 
The only extreminzer is $(\frac{5}{2}, \frac{5}{2}) $ 


\item 
\begin{align*} 
f(\vx) & = x_1^2 +6x_1x_2 -4x_1 -2x_2\\
g_1(\vx) & = x_1^2 + 2x_2 - 1 \leq 0 \\
g_2(\vx) & = 2x_1 -2x_2 - 1 \leq 0 \\
\triangledown f(\vx) &= \bmat{2x_1+6x_2-4, & 6x_1 -2} \\
\triangledown g_1(\vx) &= \bmat{2x_1, &2  } \\
\triangledown g_2(\vx) &= \bmat{2, &-2}  \\
\end{align*} 
Then we have \begin{align*} 
(2x_1 + 6x_2 -4) +\mu_1(2x_1) +2\mu_2 & = 0 \\
(6x_1-2) +2\mu_1 -2\mu_2 &= 0 \\
x_1^2 +2x_2-1 &\leq 0 \\
2x_1-2x_2-1 &\leq 0 \\
\mu_1(x_1^2 +2x_2 -1) + \mu_2 (2x_1-2x_2-1 ) &= 0 \\
\end{align*} 
If $\mu_1 = 0 $ and $\mu_2 = 0 $:   we can get $x_1=\frac{1}{3} $ and $x_2 = \frac{5}{9} $, which will make $x_1^2+2x_2-1 = \frac{2}{9} > 0  $, not feasible. \\
If $\mu_1 = 0 $ and $\mu_2 \neq 0 $:   we can get $ \bmat {\rfrac{1}{7}, & x_2 = \rfrac{9}{14}} $ \\ 
If $\mu_1 \neq 0 $ and $\mu_2 =0 $:  we can not get any solutions. \\
If $\mu_1 \neq 0 $ and $\mu_2 \neq 0 $:  we can get only one feasible point $\bmat{-1-\sqrt{2}, 3+2\sqrt{2} } $.\\ 




\end{enumerate} 

\end{document}
