\documentclass{article}

% Packages required to support encoding
\usepackage{ucs}
\usepackage[utf8x]{inputenc}
\usepackage{graphicx} 
% Packages required by code

% Packages always used
\usepackage{listings}
\usepackage{hyperref}
\usepackage{xspace}
\usepackage[usenames,dvipsnames]{color}
\hypersetup{colorlinks=true,urlcolor=blue}


\usepackage[framed,numbered,autolinebreaks,useliterate] {mcode}


\input{preamble.tex}

\title{Homework}



\begin{document} 


\subsection*{{Problem 1: }}

\subsection*{{Problem 2: }}

Using PSO algorithm, we can find the minimizer: \\
\begin{align*}
x_0 & = 0.000279321964182 \\
x_1 & = 0.000193196456243 \\
f(x_0,x_1) & =  2.28836604776e-07
\end{align*}

\begin{figure}[h]
\includegraphics[width=0.7\textwidth]{PSO} 
\centering
\caption{PSO Algorithm (problem 2): circle points are randomly generated 50 initial points. Stars indicate the positions after 50 iterations. }

\end{figure}

\begin{figure}[h]
\includegraphics[width=0.7\textwidth]{PSO_min_best} 
\centering
\caption{PSO Algorithm (problem 2): plots of the best, average, and the worst objective function values in the population for 50 generations }

\end{figure}


\subsection*{{Problem 3: }}

Using PSO algorithm, we can find the maximizer: \\
\begin{align*}
x_0 & =  -5.02482780601\\
x_1 & =  5.02524813509\\
f(x_0,x_1) & = -40.5025451078 
\end{align*}
In fact, there are several other global maximizers. PSO method will converge to different global maximizer depending on the initial points which are randomly chosen.   
\begin{figure}[h]
\includegraphics[width=0.7\textwidth]{PSO_max} 
\centering
\caption{PSO Algorithm (problem 3): circle points are randomly generated 50 initial points. Stars indicate the positions after 50 iterations. }

\end{figure}

\begin{figure}[h]
\includegraphics[width=0.7\textwidth]{PSO_max_best} 
\centering
\caption{PSO Algorithm (problem 3): plots of the best, average, and the worst objective function values in the population for 50 generations }

\end{figure}





\subsection*{{Problem 4: }}

Population size:  50  \\
Number of iterations:  50 \\ 
For canonical number genetic algorithm, the minimizer is: \\
\begin{align*}
x_1 & = 0.0408935546875 \\
x_2 & = 0.0390625  \\
f(x_1,x_2)  & = 0.00634456702034
\end{align*}

For real number genetic algorithm, the minimizer is : \\
\begin{align*} 
x_1 & = 0.018313265874 \\
x_2 & = 0.0286761643909 \\
f(x_1, x_2) & = 0.00229673023909
\end{align*} 

\begin{figure} [h]
\includegraphics[width=0.7\textwidth]{GA_canonical}
\centering
\caption{Canonical Genetic Algorithm (problem 4): circle points are randomly generated 50 initial points. Stars indicate the positions after 50 iterations. }

\includegraphics[width=0.7\textwidth]{GA_real_number}
\centering
\caption{Real Number Genetic Algorithm (problem 4): circle points are randomly generated 50 initial points. Stars indicate the positions after 50 iterations. }

\end{figure}

\begin{figure}[h]
\includegraphics[width=0.7\textwidth]{Canonical_GA_best_plot}
\centering
\caption{Canonical Genetic Algorithm (problem 4): plots of the best, average, and the worst objective function values in the population for 50 generations}


\includegraphics[width=0.7\textwidth]{Real_number_GA}
\centering
\caption{Real Number Genetic Algorithm (problem 4): plots of the best, average, and the worst objective function values in the population for 50 generations}
\end{figure}


\subsection*{{Problem 5: }}

The shortest path is shown in Figure \ref{fig:tsp1}, and the shortest distance is : $37.7222579198$

\begin{figure}[h]
\includegraphics[width=0.7\textwidth]{Traveling_salesman_path}
\centering
\caption{Traveling salesman problem (problem 5): plots of the shortest distance path}
\label{fig:tsp1}

\includegraphics[width=0.7\textwidth]{Distance_vs_generation}
\label{fig:tsp2}
\centering
\caption{Traveling salesman problem (problem 5): plots of the shortest distance for different combinations of the population for 1000 generations}

\end{figure}






\end{document}
