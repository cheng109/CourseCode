\documentclass{article}

% Packages required to support encoding
\usepackage{ucs}
\usepackage[utf8x]{inputenc}
\usepackage{graphicx} 
% Packages required by code

% Packages always used
\usepackage{listings}
\usepackage{hyperref}
\usepackage{xspace}
\usepackage[usenames,dvipsnames]{color}
\hypersetup{colorlinks=true,urlcolor=blue}


\usepackage[framed,numbered,autolinebreaks,useliterate] {mcode}


\input{preamble.tex}

\title{Homework}



\begin{document} 



\hypertarget{problem_0_homework_checklist_2}{}
\subsection*{{Problem 1}}
\label{problem_0_homework_checklist_2}

For equation $f(x) = e^{-x}-x=0 $  \newline
$f(0)=1>0$ and $f(1)=e^{-1}-1<0$.  Also function $f(x)$ is continuous monotonic decreasing function.Therefore there must be a root on the interval $(0,1) $ \newline
For the first 4 iterations.  interval [a, b] becomes: \\
\begin{itemize}
\item iter0: [0,		1	 ]\\
\item iter1: [0.5,	1	 ]\\
\item iter2: [0.5,		0.75	 ]\\
\item iter3: [0.5,		0.625]\\
\item iter4: [0.5625,	0.625]\\
\end{itemize}
Therefore $p_3=0.625$ and $(a_4, b_4)=(0.5625, 0.625) $



\hypertarget{problem_0_homework_checklist_2}{}
\subsection*{{Problem 2}}
\label{problem_0_homework_checklist_2}

For equation $f(x) = x_6-3=0 $  \newline
$f(1)=-2<0$ and $f(2)=61>0$.  Also function $f(x)$ is continuous monotonic increasing function.Therefore there must be a root on the interval $(1,2) $ \newline

Output from bisection code: \\
\begin{itemize}
\item iter0 [1,	2] \\
\item iter1 [1,	1.5]   		:actual error $ |1.5-\sqrt[6]{3}|=0.299< 0.5$ \\
\item iter2 [1,	1.25]			:actual error $ |1.25-\sqrt[6]{3}|=0.049< 0.25$ \\
\item iter3 [1.125,	1.25]		:actual error $ |1.125-\sqrt[6]{3}|=0.0759< 0.125$ \\
\item iter4 [1.1875,	1.25]		:actual error $ |1.1875-\sqrt[6]{3}|=0.0134< 0.0625$ \\
\item iter5 [1.1875,	1.21875]	:actual error $ |1.21875-\sqrt[6]{3}|=0.0178< 0..03125$ \\
\end{itemize}
We can see that each approximation satisfies the theoretical error, but the actual error does not steadily decrease. Sometimes it is large and sometimes it is small. \\

\hypertarget{problem_3_homework_checklist_2}{}
\subsection*{{Problem 3}}
\label{problem_0_homework_checklist_2}
For each step the error would become half of the interval. So 
\begin{align} 
error_n = \frac{(b-a)}{2^n}\\
\epsilon > error_n\\
\epsilon >  \frac{(b-a)}{2^n}\\
n>\log_2 \frac{(b-a)}{\epsilon}\\
\end{align}
Therefore $n$ should be the integer bigger than $\log_2 \frac{(b-a)}{\epsilon}$

\hypertarget{problem_4_homework_checklist_2}{}
\subsection*{{Problem 4}}
\label{}
\begin{enumerate}
\item 
See output of attached code.  The result is 1.73205. 
\item 
For the first 5 iterations: \\
 $|p_n-p_{n-1}|, \quad    |p_{n-1}-p|,   \quad  |p_n-p|  $\\
0.277778,  \quad   0.232051,   0.045727\\
0.0444171, \quad	0.045727,  \quad	0.00130986\\
0.00130874, \quad	0.00130986, \quad	1.12184e-06\\
1.12184e-06, \quad	1.12184e-06,\quad 8.24008e-13\\
8.23785e-13, \quad	 8.24008e-13, \quad   2.22045e-16\\
\item 
The ratios of $|p_n-p|/|p_{n-1}-p|^2$: \\
\begin{itemize}
\item 0.849193\\
\item 0.62644\\
\item 0.653856\\
\item 0.65474\\
\end{itemize}
Which is approaches to $|f''(p)/2f'(p)|=0.654701 $
\end{enumerate}


\hypertarget{problem_5_homework_checklist_2}{}
\subsection*{{Problem 5}}
\label{}
The true value is : 2.35134\\
The estimated value is 2.351
 \begin{figure}
 \centering 
 \includegraphics[width=0.5\textwidth]{root}
 \caption{Problem 5. Root vs interation number} 
 \end{figure} 
  \begin{figure}
  \centering 
 \includegraphics[width=0.5\textwidth]{error}
  \caption{Problem 5. Error vs interation number} 
 \end{figure} 


\hypertarget{problem_6_homework_checklist_2}{}
\subsection*{{Problem 6}}
\label{}
The true value (from Wolfram Alpha)  is : 1.45757030926521\\
The estimated value is 1.45757
 \begin{figure}
 \centering 
 \includegraphics[width=0.5\textwidth]{root_6}
 \caption{Problem 6. Root vs interation number} 
 \end{figure} 
  \begin{figure}
 \centering 
 \includegraphics[width=0.5\textwidth]{error_6}
  \caption{Problem 6. Error vs interation number} 
 \end{figure} 

\hypertarget{problem_6_homework_checklist_2}{}
\subsection*{{Problem 7}}
\label{}
\begin{enumerate}
\item 
$f(x)=e^x+x^2-x-4 $  \\
$x=1.28868 $ \\
\item 
$f(x) =x^3-x^2-10x+7 $ \\
$x=0.68522 $\\
\item 
$f(x) = 1.05-1.04x+ln(x) $\\
$x= 1.10971$ \\

\end{enumerate}

\end{document}


