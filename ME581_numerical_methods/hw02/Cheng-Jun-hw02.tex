\documentclass{article}

% Packages required to support encoding
\usepackage{ucs}
\usepackage[utf8x]{inputenc}
\usepackage{graphicx} 
\usepackage{enumitem}
% Packages required by code

% Packages always used
\usepackage{listings}
\usepackage{hyperref}
\usepackage{xspace}
\usepackage[usenames,dvipsnames]{color}
\hypersetup{colorlinks=true,urlcolor=blue}


\usepackage[framed,numbered,autolinebreaks,useliterate] {mcode}


\usepackage{geometry}
\geometry{letterpaper,textwidth=350pt,textheight=680pt,tmargin=60pt,
            left=72pt,footskip=24pt,headsep=18pt,headheight=14pt}
\usepackage{amsmath}
\usepackage{amssymb}
\usepackage{textcase}
\usepackage{soul}

\newcommand{\mat}[1]{\boldsymbol{#1}}\renewcommand{\vec}[1]{\boldsymbol{\mathrm{#1}}}
\newcommand{\vecalt}[1]{\boldsymbol{#1}}

\newcommand{\conj}[1]{\overline{#1}}

\newcommand{\normof}[1]{\|#1\|}
\newcommand{\onormof}[2]{\|#1\|_{#2}}

\newcommand{\itr}[2]{#1^{(#2)}}
\newcommand{\itn}[1]{^{(#1)}}

\newcommand{\eps}{\varepsilon}
\newcommand{\kron}{\otimes}

\DeclareMathOperator{\diag}{diag}
\DeclareMathOperator{\trace}{trace}
\DeclareMathOperator{\tvec}{vec}

\newcommand{\prob}{\mathbb{P}}
\newcommand{\probof}[1]{\prob\left\{ #1 \right\}}

\newcommand{\pmat}[1]{\begin{pmatrix} #1 \end{pmatrix}}
\newcommand{\bmat}[1]{\begin{bmatrix} #1 \end{bmatrix}}
\newcommand{\spmat}[1]{\left(\begin{smallmatrix} #1 \end{smallmatrix}\right)}
\newcommand{\sbmat}[1]{\left[\begin{smallmatrix} #1 \end{smallmatrix}\right]}

\newcommand{\RR}{\mathbb{R}}
\newcommand{\CC}{\mathbb{C}}

\providecommand{\eye}{\mat{I}}
\providecommand{\mA}{\ensuremath{\mat{A}}}
\providecommand{\mB}{\ensuremath{\mat{B}}}
\providecommand{\mC}{\ensuremath{\mat{C}}}
\providecommand{\mD}{\ensuremath{\mat{D}}}
\providecommand{\mE}{\ensuremath{\mat{E}}}
\providecommand{\mF}{\ensuremath{\mat{F}}}
\providecommand{\mG}{\ensuremath{\mat{G}}}
\providecommand{\mH}{\ensuremath{\mat{H}}}
\providecommand{\mI}{\ensuremath{\mat{I}}}
\providecommand{\mJ}{\ensuremath{\mat{J}}}
\providecommand{\mK}{\ensuremath{\mat{K}}}
\providecommand{\mL}{\ensuremath{\mat{L}}}
\providecommand{\mM}{\ensuremath{\mat{M}}}
\providecommand{\mN}{\ensuremath{\mat{N}}}
\providecommand{\mO}{\ensuremath{\mat{O}}}
\providecommand{\mP}{\ensuremath{\mat{P}}}
\providecommand{\mQ}{\ensuremath{\mat{Q}}}
\providecommand{\mR}{\ensuremath{\mat{R}}}
\providecommand{\mS}{\ensuremath{\mat{S}}}
\providecommand{\mT}{\ensuremath{\mat{T}}}
\providecommand{\mU}{\ensuremath{\mat{U}}}
\providecommand{\mV}{\ensuremath{\mat{V}}}
\providecommand{\mW}{\ensuremath{\mat{W}}}
\providecommand{\mX}{\ensuremath{\mat{X}}}
\providecommand{\mY}{\ensuremath{\mat{Y}}}
\providecommand{\mZ}{\ensuremath{\mat{Z}}}
\providecommand{\mLambda}{\ensuremath{\mat{\Lambda}}}
\providecommand{\mPbar}{\bar{\mP}}

\providecommand{\ones}{\vec{e}}
\providecommand{\va}{\ensuremath{\vec{a}}}
\providecommand{\vb}{\ensuremath{\vec{b}}}
\providecommand{\vc}{\ensuremath{\vec{c}}}
\providecommand{\vd}{\ensuremath{\vec{d}}}
\providecommand{\ve}{\ensuremath{\vec{e}}}
\providecommand{\vf}{\ensuremath{\vec{f}}}
\providecommand{\vg}{\ensuremath{\vec{g}}}
\providecommand{\vh}{\ensuremath{\vec{h}}}
\providecommand{\vi}{\ensuremath{\vec{i}}}
\providecommand{\vj}{\ensuremath{\vec{j}}}
\providecommand{\vk}{\ensuremath{\vec{k}}}
\providecommand{\vl}{\ensuremath{\vec{l}}}
\providecommand{\vm}{\ensuremath{\vec{l}}}
\providecommand{\vn}{\ensuremath{\vec{n}}}
\providecommand{\vo}{\ensuremath{\vec{o}}}
\providecommand{\vp}{\ensuremath{\vec{p}}}
\providecommand{\vq}{\ensuremath{\vec{q}}}
\providecommand{\vr}{\ensuremath{\vec{r}}}
\providecommand{\vs}{\ensuremath{\vec{s}}}
\providecommand{\vt}{\ensuremath{\vec{t}}}
\providecommand{\vu}{\ensuremath{\vec{u}}}
\providecommand{\vv}{\ensuremath{\vec{v}}}
\providecommand{\vw}{\ensuremath{\vec{w}}}
\providecommand{\vx}{\ensuremath{\vec{x}}}
\providecommand{\vy}{\ensuremath{\vec{y}}}
\providecommand{\vz}{\ensuremath{\vec{z}}}
\providecommand{\vpi}{\ensuremath{\vecalt{\pi}}}

\sodef\allcapsspacing{\upshape}{0.15em}{0.65em}{0.6em}%

\makeatletter
\def\maketitle{%
\par
\hrule height 0.75pt\vspace{1ex}
\par\noindent
\begin{minipage}{0.5\textwidth}
\scshape
purdue university $\cdot$ CS 580 \\
Introduction to the Analysis of Algorithms
\end{minipage}
\begin{minipage}{0.5\textwidth}
\raggedleft
\MakeTextUppercase{\allcapsspacing{\@title}}\\[0.2ex]
\textit{\@author}\\[0.2ex]
\textit{\@date}
\end{minipage}
\par\vspace{1ex}
\hrule height 1pt
\vspace{2ex}
\par
}
\makeatother

\author{Jun Cheng}
\title{Lecture Notes}
% auto generate a title
\AtBeginDocument{\maketitle}


\title{Homework}



\begin{document} 



\hypertarget{problem_0_homework_checklist_2}{}
\subsection*{{Problem 4: }}
\label{}

\begin{enumerate}[label=(\alph*)]
\item 
\begin{align*} 
\mA &= \bmat{0 & 1 & 1 & 1\\ 3 & 0 & 3 & -4\\ 1 & 1 & 1 & 2\\ 2 & 3 & 1 & 3}\\
\vb &= \bmat{0 \\ 7 \\ 6 \\ 6 } 
\end{align*} 

\begin{enumerate}[label=(\roman*)] 
\item Partial pivoting 
The output from my code is \begin{align*} \vx = \bmat{4 \\ -3 \\ 1 \\ 2 } \end{align*} 
\item Scaled pivoting 
The output from my code is \begin{align*} \vx = \bmat{4 \\ -3 \\ 1 \\ 2 } \end{align*} 
\end{enumerate} 


\item 
\begin{align*} 
\mA &= \bmat{	0.2115 & 2.296 & 2.715 & 3.215 \\ 
			0.4371 & 3.916 & 1.683 & 2.852 \\ 
			6.099 & 4.324 & 23.20 & 1.578 \\
			4.623 & 0.8926 & 15.32 & 5.305 
			}\\
\vb &= \bmat{8.438 \\ 8.8888 \\ 35.20 \\ 26.14} 
\end{align*} 



\begin{enumerate}[label=(\roman*)] 
\item Partial pivoting 
The output from my code is \begin{align*} \vx = \bmat{0.999081 \\ 0.999913 \\ 1.00021 \\ 1.0001 } \end{align*} 
\item Scaled pivoting 
The output from my code is \begin{align*} \vx = \bmat{0.999081 \\ 0.999913 \\ 1.00021 \\ 1.0001 } \end{align*} 
\end{enumerate} 
\end{enumerate} 

\hypertarget{problem_0_homework_checklist_2}{}
\subsection*{{Problem 5: }}
\label{}
\begin{align*}
\mA &= \bmat{-9 & 11 & -21 & 63 & -252 \\ 
			70 & -69 & 141 & - 421 & 1684 \\
			-575 & 575 & -1149 & 3451 & -13801 \\
			3891 & -3891 & 7782 & -23345 & 93365 \\
			1024 & -1024 & 2048 & -6144 & 24572 }\\
\vb & = \bmat{-356 \\ 2385 \\ -19551 \\ 132274 \\ 34812 }
\end{align*} 
The output of scaled pivoting result is 
\begin{align*}
 \widetilde\vx &= \bmat{1 \\ -0.864667 \\ 0.0848896 \\ 5.27044 \\ 2.64978} \\
 \ve &= \widetilde\vx - \vx  = \bmat{0  \\  0.1353  \\ -0.9151 \\   6.2704    \\1.6498 } \\
\frac{ \|\ve\|}{\|\vx\|} &= 2.929\\
\mA\widetilde\vx-\vb &= \bmat{-0.0009\\   0.0057\\   -0.0470\\    0.3181\\    0.0837}\\
\frac{ \|\vr\|}{\|\vb\|} & = \frac{ \|\mA\widetilde\vx-\vb\|}{\|\vb\|}= \frac{0.3323}{1.3819e+05}=2.4046e-06\\
 \end{align*} 

Therefore, the estimated condition number is $\frac{2.929}{2.4046e-06} = 1.2181e+06$



\hypertarget{problem_0_homework_checklist_2}{}
\subsection*{{Problem 6: }}
\label{}

We can construct 12 equation if there are no external forces, 2 for each node: 
\begin{align*}
\sum F_{1x} &=F_1\cos\alpha + F_3\cos\gamma+F_H &= 0 \\
\sum F_{1y} &=F_v + F_3\cos\gamma + F_1\sin\alpha &= 0 \\
\sum F_{2x} &=F_8\cos\beta + F_9 \cos\alpha-F_2\cos\beta-F_1\cos\alpha &= 0 \\
\sum F_{2y} &=F_5+F_2\sin\beta+F_8\sin\beta-F_1\sin\alpha-F_9\sin\alpha &= 0 \\
\sum F_{3x} &=F_4\cos\beta+F_2\cos\beta-F_3\cos\gamma &= 0 \\
\sum F_{3y} &=F_4\sin\beta-F_2\sin\beta-F_3\sin\gamma &= 0 \\
\sum F_{4x} &=F_6\cos\beta - F_4\cos\beta &= 0 \\
\sum F_{4y} &=-F_4\sin\beta - F_6\sin\beta-F_5 &=0 \\
\sum F_{5x} &=F_7\cos\gamma - F_6\cos\beta -F_8\cos\beta &= 0 \\
\sum F_{5y} &=F_6\sin\beta-F_8\sin\beta-F_7\sin\gamma &= 0 \\
\sum F_{6x} &=-F_7\cos\gamma-F_9\cos\alpha &= 0 \\
\sum F_{6y} &=F_R+F_7\sin\gamma+F_9\sin\alpha &= 0 \\
\end{align*} 
Then the linear equation $\mA\vx=\vb$ becomes: \\
\begin{align*}
\left[\begin{array}{cccccccccccc}
\cos\alpha & 0 & \cos\gamma & 0 & 0 & 0 & 0 & 0 & 0 & 0 & 0 & 1 \\ 
\sin\alpha & 0 & \cos\gamma & 0 & 0 & 0 & 0 & 0 & 0 & 0 & 1 & 0 \\ 
-\cos\alpha & -\cos\beta & 0 & 0 & 0 & 0 & 0 & \cos\beta & \cos\alpha & 0 & 0 & 0 \\ 
-\sin\alpha & \sin\beta & 0 & 0 & 1 & 0 & 0 & \sin\beta & -\sin\alpha & 0 & 0 & 0 \\ 
0 & \cos\beta & -\cos\gamma & \cos\beta & 0 & 0 & 0 & 0 & 0 & 0 & 0 & 0 \\ 
0 & -\sin\beta & -\sin\gamma & \sin\beta & 0 & 0 & 0 & 0 & 0 & 0 & 0 & 0 \\ 
0 & 0 & 0 & -\cos\beta & 0 & \cos\beta & 0 & 0 & 0 & 0 & 0 & 0 \\ 
0 & 0 & 0 & -\sin\beta & -1 & -\sin\beta & 0 & 0 & 0 & 0 & 0 & 0 \\ 
0 & 0 & 0 & 0 & 0 & -\cos\beta & \cos\gamma & -\cos\beta & 0 & 0 & 0 & 0 \\ 
0 & 0 & 0 & 0 & 0 & \sin\beta & -\sin\gamma & -\sin\beta & 0 & 0 & 0 & 0 \\ 
0 & 0 & 0 & 0 & 0 & 0 & -\cos\gamma & 0 & \cos\alpha & 0 & 0 & 0 \\
0 & 0 & 0 & 0 & 0 & 0 & \sin\gamma & 0 & \sin\alpha & 1 & 0 & 0 \\  
\end{array}\right]
\bmat{F_1 \\ F_2 \\ F_3 \\ F_4 \\ F_5 \\ F_6 \\ F_7 \\ F_8 \\ F_9 \\ F_R \\ F_V \\ F_H }=\vb
\end{align*} 
We know: \\
\begin{align*}
\sin\alpha = 0.447 \\
\cos\alpha = 0.894 \\
\sin\beta = 0.316\\
\cos\beta =  0.949\\
\sin\gamma = 0.707 \\ 
\cos\gamma =0.707 \\ 
\end{align*} 
Then matrix A becomes: \\ 
\begin{align*}
\mA = \left[\begin{array}{cccccccccccc}
0.894 & 0 & 0.707 & 0 & 0 & 0 & 0 & 0 & 0 & 0 & 0 & 1 \\ 
0.447 & 0 & 0.707 & 0 & 0 & 0 & 0 & 0 & 0 & 0 & 1 & 0 \\ 
-0.894 & -0.949 & 0 & 0 & 0 & 0 & 0 & 0.949 & 0.894 & 0 & 0 & 0 \\ 
-0.447 & 0.316 & 0 & 0 & 1 & 0 & 0 & 0.316 & -0.316 & 0 & 0 & 0 \\ 
0 & 0.949 & -0.707 & 0.949 & 0 & 0 & 0 & 0 & 0 & 0 & 0 & 0 \\ 
0 & -0.316 & -0.707 & 0.316 & 0 & 0 & 0 & 0 & 0 & 0 & 0 & 0 \\ 
0 & 0 & 0 & -0.949 & 0 & 0.949 & 0 & 0 & 0 & 0 & 0 & 0 \\ 
0 & 0 & 0 & -0.316 & -1 & -0.316 & 0 & 0 & 0 & 0 & 0 & 0 \\ 
0 & 0 & 0 & 0 & 0 & -0.949 & 0.707 & -0.949 & 0 & 0 & 0 & 0 \\ 
0 & 0 & 0 & 0 & 0 & 0.316 & -0.707 & -0.316 & 0 & 0 & 0 & 0 \\ 
0 & 0 & 0 & 0 & 0 & 0 & -0.707 & 0 & 0.894 & 0 & 0 & 0 \\
0 & 0 & 0 & 0 & 0 & 0 & 0.707 & 0 & 0.447 & 1 & 0 & 0 \\  
\end{array}\right]
\end{align*} 

For different configurations: \\
\begin{align*}
\vb_a = \bmat{0 \\ 0 \\ 0 \\ 0 \\ 0 \\ 500 \\  0 \\ 1000 \\ 0 \\ 500 \\ 0 \\ 0}\ \ \ \ \ 
\vb_b = \bmat{0 \\ 0 \\ 0 \\ 0 \\ 0 \\ 500 \\  0 \\ 1000 \\ 0 \\ 1500 \\ 0 \\ 0}\ \ \ \ \ 
\vb_c = \bmat{0 \\ 0 \\ 0 \\ 0 \\ 0 \\ 1500 \\  0 \\ 1000 \\ 0 \\ 500 \\ 0 \\ 0}\ \ \ \ \ 
\vb_d = \bmat{0 \\ 0 \\ 0 \\ 0 \\ -1000 \\ 0 \\  -500 \\ 0 \\ 0 \\ 0 \\ 0 \\ 0}\ \ \ \ \ 
\vb_e = \bmat{0 \\ 0 \\ 0 \\ 0 \\ 0 \\ 0 \\  500 \\ 0 \\ 1000 \\ 0 \\ 0 \\ 0}\ \ \ \ \ 
\end{align*}
The results are shown: \\
\begin{align*}
F_a= \bmat{	-784.006 \\ -51.402 \\ -991.374 \\ -687.167 \\ -565.711 \\ -687.167 \\
			-991.374 \\ -51.402 \\ -784.006 \\ 1051.35 \\ 1051.35 \\1401.8 \\ }
F_b= \bmat{	-2162.2 \\ -334.561 \\ -611.893 \\ -121.296 \\ -923.341 \\ -121.296 \\ 
			-1672.99 \\ -1125.07 \\ -1323.05 \\ 1774.21 \\ 1399.11 \\ 2365.61 \\}
F_c=\bmat{	-189.822 \\ -610.785 \\ -2362.23 \\ -1149.06 \\ -273.791 \\ -1149.06 \\
			 -1301.13 \\ 179.729 \\ -1028.97 \\ 1379.85 \\ 1754.95 \\ 1839.8 \\}
F_d=\bmat{	727.085 \\ -684.362 \\ 211.065 \\ -212.136 \\ 300.561 \\ -739.007 \\ 
			-495.589 \\ 369.796 \\ -391.926 \\ 525.572 \\ -474.23 \\ -799.237 \\}
F_e=\bmat{	-727.085 \\ 157.491 \\ -211.065 \\ -314.734 \\ 32.4208 \\ 212.136 \\ 
			495.589 \\ -896.666 \\ 391.926 \\ -525.572 \\ 474.23 \\ 799.237 \\ }
\end{align*} 
For the direction of the forces:  negative values indicate forces point outward. Positive direction of $F_V, F_H, F_R $ are as shown in the figure. 
\end{document}
