\documentclass{article}

% Packages required to support encoding
\usepackage{ucs}
\usepackage[utf8x]{inputenc}
\usepackage{graphicx} 
% Packages required by code

% Packages always used
\usepackage{listings}
\usepackage{hyperref}
\usepackage{xspace}
\usepackage[usenames,dvipsnames]{color}
\hypersetup{colorlinks=true,urlcolor=blue}


\usepackage[framed,numbered,autolinebreaks,useliterate] {mcode}


\input{preamble.tex}

\title{Homework}



\begin{document} 



\hypertarget{problem_0_homework_checklist_2}{}
\subsection*{{Problem 1: Gauss Seidel Method}}
\label{}
\begin{align} 
\mA &= \bmat{5 & -1 & 0 & -1 & 0 & 0  \\ 
-1 & 5 & -1 & 0 & -1 & 0  \\ 
0 & -1 & 5 & 0 & 0 & -1  \\ 
-1 & 0 & 0 & 5 & -1 & 0  \\ 
0 & -1 & 0 & -1 & 5 & -1  \\ 
0 & 0 & -1 & 0 & -1 & 5  \\ }\\
\vb &=\bmat{-5 \\ 1\\ 1\\ -2 \\ 1 \\ 2 } \\
\vx_0 &=\bmat{0 \\ 0 \\ 0 \\ 0 \\ 0 }
\end{align} 
For the first 10 iterations:  \\
residual: 1.27057 \\
residual: 0.189021 \\
residual: 0.0518507 \\
residual: 0.0175355 \\
residual: 0.00437263 \\
residual: 0.00103081 \\
residual: 0.000240774 \\
residual: 5.61515e-05 \\
residual: 1.30917e-05 \\
residual: 3.0522e-06 \\

The residual is defined as   $\|x_{n+1}-x_{n}\|_2$ 

\begin{align}
\vx =\bmat{-1.09909\\
0.0840332\\
0.317577\\
-0.579482\\
0.20168\\
0.503851}
\end{align} 




\end{document}
